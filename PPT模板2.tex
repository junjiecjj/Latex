% 使用 xelatex 编译
\documentclass[8pt,compress,t,notheorems,noamsthm,notheorem,xcolor=x11names]{beamer}
%\documentclass[8pt,compress,t,handout,table]{beamer}
\usepackage{xeCJK}
\setCJKmainfont{SimSun}

%========================== 演示主题================================================
%◦ 没有导航条:default,AnnArbor,Bergen,Boadilla,CambridgeUS,EastLansing,Madrid,Pittsburgh,Rochester.
%◦ 带树形导航条: Antibes, JuanLesPins, Montpellier
%◦ 带侧边导航条: Berkeley, Goettingen, Hannover, Marburg, PaloAlto
%◦ 带微型导航条: Berlin, Ilmenau, Darmstadt, Dresden, Frankfurt,Singapore, Szeged
%◦ 带节小节标题: Copenhagen, Luebeck, Malmoe, Warsaw
\usetheme{Boadilla}

%========================== 颜色主题============================================
%- Default Color Theme: default
%- Special-Purpose Color Themes: structure, sidebartab
%- Complete Color Themes:albatross,beetle,crane,dove,fly,seagull,wolverine,beaver,spruce
%- Inner Color Themes: lily, orchid, rose
%- Outer Color Themes: whale, seahorse, dolphin,

\usecolortheme{seahorse}
\usecolortheme[named=SpringGreen4]{structure}
\usecolortheme{rose}
\usecolortheme{beaver}
%========================== 字体主题========================================
%default,professionalfonts,serif,structurebold,structureitalicserif,structuresmallcapsserif

\usefonttheme{structurebold}
\usefonttheme[onlymath]{serif}

%========================== 外部主题========================================
%default, infolines, miniframes, smoothbars, sidebar, split, shadow, tree, smoothtree
\useoutertheme[subsection=false]{smoothbars}
%\useoutertheme[subsection=false]{miniframes}

%========================== 内部主题=====================================
%default, circles, rectangles, rounded, inmargin
%\useinnertheme[shadow]{rounded} 
\useinnertheme{circles}
%%%%%%%%%%%%%%%%%%%%%%%%%%%%%%%%%%%%%%%%%%%%%%%%%%%%%%%%%%%%%%%%

\setbeamercolor{separation line}{use=structure,bg=structure.fg!50!bg}
%\setbeamertemplate{navigation symbols}{}
\setbeamersize{text margin left=0.5cm, text margin right=0.5cm}
\setbeamerfont{frametitle}{size=\normalsize}
\definecolor{myfoot}{rgb}{0.5,0.2,0.5}
\definecolor{darkblue}{rgb}{0.1,0,0.85}


%\setbeamerfont{frametitle}{size=\large,}
%\setbeamerfont{headline}{size=\footnotesize,}
%\setbeamerfont{footline}{size=\large,}

\setbeamertemplate{title page}
{
	%  \vbox{}
	\vfill
	\begin{centering}
		\vskip0.6em\par%
		\includegraphics[height=2.0cm]{ustc_logo_fig_new.eps}
		\vskip0.4em\par%
		\begin{beamercolorbox}[sep=8pt,center,shadow=true,rounded=true]{title}
			\usebeamerfont{title}\inserttitle\par%
			\ifx\insertsubtitle\@empty%
			\else%
			\vskip0.25em%
			{\usebeamerfont{subtitle}\usebeamercolor[fg]{subtitle}\insertsubtitle\par}%
			\fi%
		\end{beamercolorbox}%
		\vskip1em\par
		\begin{beamercolorbox}[sep=8pt,center]{author}
			\usebeamerfont{author}
			%     \insertauthor
			{
				\begin{tabular}{cc}
					报告人:& \insertauthor\\
					导\quad 师:& \advisor
				\end{tabular}
			}
		\end{beamercolorbox}
		\begin{beamercolorbox}[sep=10pt,center]{institute}
			\usebeamerfont{institute}\insertinstitute
		\end{beamercolorbox}
		\begin{beamercolorbox}[sep=10pt,center]{date}
			\usebeamerfont{date}\insertdate
		\end{beamercolorbox}\vskip1.5em
		%  {\usebeamercolor[fg]{titlegraphic}\inserttitlegraphic\par}
	\end{centering}
	\vfill
}

\setbeamertemplate{footline}% 自定义页脚
{
	\leavevmode\mbox{%
    \begin{beamercolorbox}[wd=.75\paperwidth,ht=2.25ex,dp=1ex,left]{myfootline}%
        \rule{2em}{0pt}\color{myfoot}\ttfamily\scriptsize
        \insertshortauthor~(\insertshortinstitute)
    \end{beamercolorbox}%
    \begin{beamercolorbox}[wd=.25\paperwidth,ht=2.25ex,dp=1ex,right]{myfootline}%
    	{
       		\color{myfoot}\ttfamily\scriptsize\insertframenumber{}/%
        	\inserttotalframenumber\hspace*{3ex}
        }
    	\end{beamercolorbox}
    	}
    	\vskip0pt
}

\setbeamercolor{frametitle}{fg=blue,bg=white}
\setbeamertemplate{frametitle}
{%
	\leavevmode\linespread{1}\normalsize{\insertframetitle}\par %控制字体大小
	\color{green!80}\rule[8pt]{\linewidth}{1.5pt}\par\vspace{-1.0em}
}

\setbeamersize{text margin left=0.05\textwidth, text margin right=1cm,}
\setbeamercolor{bluebox}{fg=black,bg=blue!10}
\setbeamercolor{redbox}{fg=black,bg=red!10}
\newenvironment{Boxblue}[1][\textwidth]
	{\begin{beamercolorbox}[sep=0.1em,shadow=true,wd=#1,rounded=true,center]{bluebox}}
	{\end{beamercolorbox}}
\newenvironment{Boxred}[1][\textwidth]
	{\begin{beamercolorbox}[sep=0.1em,shadow=true,wd=#1,rounded=true,center]{redbox}}
	{\end{beamercolorbox}}

%%%%% ===== 中英文字体=============================================
\usepackage[noindent]{ctex}
\setmainfont{Times New Roman}
\setsansfont{Times New Roman} % 无衬线字体 sans serif \sffamily
\setmonofont{WenQuanYi Micro Hei}   % 等宽字体 typewriter \ttfamily
%% 中文字体
\setCJKmainfont[BoldFont={SimSun}, ItalicFont={SimSun}]{SimSun}	%设置缺省中文字体,根据自己系统里的字体选择合适字体
\setCJKsansfont{WenQuanYi Micro Hei}%SimHei, WenQuanYi Micro Hei
\setCJKmonofont{WenQuanYi Micro Hei}

%%%%% ===== 常用宏包 =======================================================
\usepackage{amsmath,amssymb,amsfonts,bm}
\usepackage{graphicx}
  \graphicspath{{figures/}}
\usepackage{xcolor}
\usepackage{hyperref}
  \hypersetup{pdfborder=001,colorlinks=true,linkcolor=darkblue,urlcolor=blue,citecolor=blue,bookmarksopen=true, breaklinks=true}
\usepackage{bbding}


\usepackage{multimedia}	%让文档支持多媒体
\usepackage{graphics}	%让文档支持图片
\usepackage{hyperref}	%让文档支持超链接
\usepackage{booktabs}	%让文档支持三线表格
\usepackage{amsmath}	%ams可以让文档支持数学公式
\usepackage{amsfonts}
\usepackage{amssymb}
\usepackage{color}
\usepackage{graphicx,psfrag}
\usepackage{epsfig}

%%%%% ===== 自加宏包 ========================================================
\usepackage{multirow}
\usepackage{amstext}
\usepackage{verbatim}%多行注释
\usepackage{fontspec}   %修改字体

\usepackage{cite}%连续引用多个文献
\usepackage{CJKpunct}

\usepackage{float} %设置图片浮动位置的宏包
\usepackage{stfloats} %设置图片浮动位置的宏包
\usepackage{subfigure} %插入多图时用子图显示的宏包



\usepackage{caption}
\usepackage{lipsum}
\usepackage{multicol}%分栏宏
\usepackage{supertabular}

\vspace{0.0cm}  %调整图片与上文的垂直距离,调整表与上文的垂直距离
\setlength{\abovecaptionskip}{0.1cm}  %调整图片标题与图距离,调整表标题与表距离
\setlength{\belowcaptionskip}{0.0cm}   %调整图片标题与下文距离,调整表标题与下文距离
%%%%% ===== 自加宏包 ========================================================

%%%%% ===== 自定义列表 ======================================================
\newcommand{\Bullet}{{\fontsize{6pt}{6pt}\selectfont\CircleSolid}}
\newcommand{\Hand}{{\fontsize{8pt}{6pt}\selectfont\HandRight}}
\newcommand{\zhu}{{\color{blue!40}\Bullet}}
\newcommand{\zhuu}{{\color{red!80}\Hand}}
\newcommand{\labeli}{\zhu}
\newenvironment{myitem}
{
	\begin{list}
		{{\hfill\raisebox{0pt}{\labeli}}}{%
    	\setlength{\leftmargin}{1.2em}\labelwidth0.8em\labelsep.4em%
    	\itemsep1ex\parsep2pt\itemindent0pt\topsep0pt}}
	{\end{list}
}
\newenvironment{subitem}
{
	\begin{list}{{\hfill\raisebox{0pt}{\zhuu}}}{%
    \setlength{\leftmargin}{1.2em}\labelwidth0.8em\labelsep.4em%
    \itemsep0ex\parsep2pt\itemindent0pt\topsep0pt}}
	{\end{list}
}
\usepackage{colortbl}
\usepackage{booktabs}
\usepackage{tikz}
\usetikzlibrary{arrows}
\usepackage{algorithm,algpseudocode,caption}
\usepackage[framemethod=tikz]{mdframed}
\newmdenv[linecolor=green,middlelinewidth=1pt,%
		  roundcorner=3pt,backgroundcolor=yellow!5,%
		  innertopmargin=0.5em,innerbottommargin=0.5em,%
		  innerleftmargin=3pt,innerrightmargin=3pt,%
		  skipbelow=0.5em,skipabove=1em,%
		  splittopskip=\topskip]{Block}
\newmdenv[linecolor=green,middlelinewidth=0.5pt,%
		  outerlinewidth=0.5pt,skipabove=0pt,
		  roundcorner=3pt,backgroundcolor=white,%
		  innerbottommargin=3pt,innerrightmargin=5pt,%
		  innerleftmargin=5pt,leftmargin=0ex]{mathbox}
\newmdenv[linecolor=blue!5!green,middlelinewidth=0.5pt,%
		  roundcorner=3pt,backgroundcolor=yellow!5,%
		  frametitle={Hello},frametitlebackgroundcolor=green!50,%
		  skipabove=2pt,skipbelow=2pt,%
		  innerleftmargin=3pt,leftmargin=0ex]{notebox}
\newmdenv[linecolor=white,font={\scriptsize},%
		  fontcolor=blue!85,backgroundcolor=yellow!5,%
		  skipabove=1ex,skipbelow=0pt,innerbottommargin=0.5ex,%
	      innerleftmargin=3pt,leftmargin=1em]{myref}

%%%%%%%%%%%%%%%%%%%%%%%%%%%%%%%%%%%%%%%%%%%%%%%%%%%%%%%%%%%%%%%%%%%%%%%%%%%%%%
\renewcommand{\thefootnote}{}% 不要编号
\setbeamertemplate{footnote}{% 首行不缩进
	\noindent\insertfootnotemark%
	\scriptsize\color{blue!85!green!85}\insertfootnotetext\par\kern1ex}
\renewcommand\footnoterule%    更改横线属性:长度,粗细,颜色
  {\color{red}\kern-3pt\rule{0.4\linewidth}{0.5pt}\par\kern2.6pt}
%%%%%%%%%%%%%%%%%%%%%%%%%%%%%%%%%%%%%%%%%%%%%%%%%%%%%%%%%%%%%%%%%%%%%%%%%%%%%%

%%%%%===== 定理环境 ===========================================
\usepackage[amsmath,thref,thmmarks,hyperref]{ntheorem}
\theorempreskipamount1.2em  % spacing before the environment
\theorempostskipamount0em % spacing after the environment
%\theorempostwork{\noindent}
\theoremstyle{nonumberplain}%{nonumberbreak}
\theoremheaderfont{\color{blue}\upshape}
%\theorembodyfont{\kaishu\color{black}}
\theoremindent0em
\theoremseparator{:\hspace{0.2em}}
%\theoremnumbering{arabic}
\colorlet{thmcolor}{gray!40}
\newmdtheoremenv[linecolor=thmcolor,middlelinewidth=1pt,
    roundcorner=3pt,backgroundcolor=white,%
    innertopmargin=0.5em,innerbottommargin=0.5em,%
    innerleftmargin=3pt,innerrightmargin=3pt,%
    skipbelow=0.5em,skipabove=1em,%
    splittopskip=\topskip,ntheorem]{theorem}%
    {定理}
\newmdtheoremenv[linecolor=thmcolor,middlelinewidth=0.5pt,
    roundcorner=3pt,backgroundcolor=white,%
    innertopmargin=0.5em,innerbottommargin=0.5em,%
    innerleftmargin=3pt,innerrightmargin=3pt,%
    skipbelow=0.5em,skipabove=1em,%
    splittopskip=\topskip,ntheorem]{corollary}%
    {推论}
\newmdtheoremenv[linecolor=thmcolor,middlelinewidth=0.5pt,
    roundcorner=3pt,backgroundcolor=white,%
    innertopmargin=0.5em,innerbottommargin=0.5em,%
    innerleftmargin=3pt,innerrightmargin=3pt,%
    skipbelow=0.5em,skipabove=1em,%
    splittopskip=\topskip,ntheorem]{lemma}%
    {引理}
%\newmdtheoremenv[linecolor=thmcolor,middlelinewidth=0.5pt,
%    roundcorner=3pt,backgroundcolor=white,%
%    innertopmargin=0.5em,innerbottommargin=0.5em,%
%    innerleftmargin=3pt,innerrightmargin=3pt,%
%    skipbelow=0.5em,skipabove=1em,%
%    splittopskip=\topskip,ntheorem]{definition}%
%    {定义}
\newmdenv[innertopmargin=0pt,roundcorner=5pt,linewidth=2pt,
    linecolor=gray!40,
    innertopmargin=1.2em,innerleftmargin=1ex,innerrightmargin=1ex,
    singleextra={
      \node[anchor=west,xshift=1em,fill=gray!20,rounded corners=4pt,%
       draw=green!50,line width=1pt]%
       at (P-|O) {\ \color{blue} 定\ 义\ \mbox{}};
    }]{definition}

\theoremstyle{plain}
\newmdtheoremenv[linecolor=thmcolor,middlelinewidth=0.5pt,
    roundcorner=3pt,backgroundcolor=white,%
    innertopmargin=0.5em,innerbottommargin=0.5em,%
    innerleftmargin=3pt,innerrightmargin=3pt,%
    skipbelow=0.5em,skipabove=1em,%
    splittopskip=\topskip,ntheorem]{example}%
    {例}
    
\newenvironment{proof}[1][证明]%
  {\par\noindent\normalfont{\hei\color{blue} #1.} \upshape}
  {\mbox{}\hfill\scalebox{1.2}{\ensuremath{\Box}}\medskip}

\usepackage[ntheorem]{empheq}
\usepackage[many]{tcolorbox}
\tcbset{highlight math %
	style={enhanced, colframe=blue!40,colback=yellow!20,arc=4pt,boxrule=1pt}}
\newtcbox{\tcbhighmathinner}[1][]{nobeforeafter,math upper,tcbox raise base,
	enhanced,left=1ex,right=1ex,top=1ex,bottom=1ex,boxsep=0mm,
	colframe=green,colback=blue!5!white,arc=4pt,boxrule=1pt,#1}

\linespread{1.3}
\setlength{\parskip}{1ex}

%%%%% ===== 自定义命令 =====================================================
\newcommand{\bbm}{\begin{bmatrix}}
\newcommand{\ebm}{\end{bmatrix}}
\newcommand{\beq}{\begin{equation}}
\newcommand{\eeq}{\end{equation}}
\renewcommand{\C}{\mathbb{C}}
\newcommand{\R}{\mathbb{R}}
\newcommand{\Rn}{\mathbb{R}^{n\times n}}
\newcommand{\IA}{\mathcal{A}}
\newcommand{\IK}{\mathcal{K}}
\newcommand{\IO}{\mathcal{O}}
\newcommand{\lam}{\lambda}
\newcommand{\eps}{\varepsilon}
\newcommand{\dis}{\displaystyle}
\newcommand{\mycite}[1]{\textcolor{blue!50!white}{\upshape{#1}}}
\newcommand{\Der}{\,\mathrm{D}}
\newcommand{\der}{\,\mathrm{d}}
%
\newcommand{\myem}[1]{\textcolor{blue}{#1}}
\DeclareMathOperator{\diag}{diag}
\DeclareMathOperator{\rank}{rank}

\begin{document}

\title[短标题]{{\huge E惺惺惜惺惺寻寻寻寻寻寻寻}}
% \subtitle{也可以有个副标题}

\author[陈俊杰]{\zihao{-5} 陈俊杰}

\institute[中国科大]{\zihao{-5}中国科学技术大学 \quad 工程与应用物理系}
\def\advisor{\zihao{-5} 休息休息}
\date{2019.09.11}

\begin{frame}[plain]
  \titlepage
\end{frame}

\begin{frame}{内容提要}
  %\frametitle{内容提要}
  \tableofcontents[hideallsubsections]
\end{frame}


% 在每节前插入目录
\AtBeginSection[]
{
	\begin{frame}<beamer>
	\frametitle{报告提要}
	\tableofcontents[currentsection,hideallsubsections]
	\end{frame}
}


\section{背景介绍}

\begin{frame}
  \frametitle{背景介绍}
\begin{myitem}
\item 考虑问题
    $$ a^2+b^2=c^2.$$

%\bigskip
\item 问题应用背景
    \begin{subitem}
        \item  xxxxx
        \item  xxxx
        \item  xxxxx
        \item $\cdots\ \cdots$
    \end{subitem}
\end{myitem}
\end{frame}


%beamer 自动加载 amsthm 宏包, 并定义如下定理环境:theorem, lemma, corollary, definition, proof, ...

	%theorem 定理环境
	%lemma 引理环境
	%proof 证明环境
	%corollary 推论环境
	%example 示例环境


\section{定义与定理}
\begin{frame}{定义与定理}

  \begin{definition} 
    这是连续的定义, 这是连续的定义, 这是连续的定义,
    这是连续的定义, 这是连续的定义, 这是连续的定义,
    这是连续的定义, 这是连续的定义, 这是连续的定义.sd少时诵诗书所所所所所所所所所所所所所所所所所所所少时诵诗书所所所所所所所所所所所所所所所所所所
  \end{definition}
\end{frame}

\begin{frame}{定义与定理 (续)}
  \begin{theorem}[中值定理]
    这是中值定理, 这是中值定理, 这是中值定理, 这是中值定理,
    这是中值定理, 这是中值定理, 这是中值定理, 这是中值定理,
    这是中值定理, 这是中值定理, 这是中值定理, 这是中值定理,
    这是中值定理, 这是中值定理, 这是中值定理, 这是中值定理.
  \end{theorem}
\end{frame}


\begin{frame}{定义与定理 (续)}
\begin{corollary}[推论1]
	这是中值定理, 这是中值定理, 这是中值定理, 这是中值定理,
	这是中值定理, 这是中值定理, 这是中值定理, 这是中值定理,
	这是中值定理, 这是中值定理, 这是中值定理, 这是中值定理,
	这是中值定理, 这是中值定理, 这是中值定理, 这是中值定理.
\end{corollary}
\end{frame}

\begin{frame}{定义与定理 (续)}
\begin{lemma}[引理1]
	这是中值定理, 这是中值定理, 这是中值定理, 这是中值定理,
	这是中值定理, 这是中值定理, 这是中值定理, 这是中值定理,
	这是中值定理, 这是中值定理, 这是中值定理, 这是中值定理,
	这是中值定理, 这是中值定理, 这是中值定理, 这是中值定理.
\end{lemma}
\end{frame}

\begin{frame}{定义与定理 (续)}
\begin{example}[例一]
	这是中值定理, 这是中值定理, 这是中值定理, 这是中值定理,
	这是中值定理, 这是中值定理, 这是中值定理, 这是中值定理,
	这是中值定理, 这是中值定理, 这是中值定理, 这是中值定理,
	这是中值定理, 这是中值定理, 这是中值定理, 这是中值定理.
\end{example}
\end{frame}

\begin{frame}{定义与定理 (续)}
\begin{example}[例二]
	这是中值定理, 这是中值定理, 这是中值定理, 这是中值定理,
	这是中值定理, 这是中值定理, 这是中值定理, 这是中值定理,
	这是中值定理, 这是中值定理, 这是中值定理, 这是中值定理,
	这是中值定理, 这是中值定理, 这是中值定理, 这是中值定理.
\end{example}
\end{frame}


\begin{frame}{定义与定理 (续)}
\begin{lemma}
	这是中值定理, 这是中值定理, 这是中值定理, 这是中值定理,
	这是中值定理, 这是中值定理, 这是中值定理, 这是中值定理,
	这是中值定理, 这是中值定理, 这是中值定理, 这是中值定理,
	这是中值定理, 这是中值定理, 这是中值定理, 这是中值定理.
\end{lemma}
\end{frame}

\section{算法描述}
\begin{frame}{算法描述}
\begin{myitem}
  \item 基本思想
  \begin{subitem}
    \item xxxxxx
    \item xxxxxx
  \end{subitem}
  \bigskip

  \item 主要优点
  \begin{subitem}
    \item  xxxx
    \item  xxxxx
  \end{subitem}
\end{myitem}
\end{frame}


\begin{frame}{算法 1}

  算法 1

\end{frame}

\begin{frame}
\frametitle{Examples: One-Dimensional Problem \mycite{BN06}}

\begin{mathbox}
	$K=[t_{ij}]\in\R^{n\times n}$ is a Toeplitz matrix defined by
	\begin{align*}
	\text{(i)}\quad
	& t_{ij}=\frac{1}{\sqrt{|i-j|}+1}
	&\to&\ \textcolor{blue}{\textit{well-conditioned}} \\
	\text{(ii)}\quad
	& t_{ij}=\dfrac{1}{\sqrt{2\pi}\sigma}e^{\frac{-|i-j|^2}{2\sigma^2}}
	\ \text{with}\ \sigma=2
	&\to&\ \textcolor{blue}{ill-conditioned}
	\end{align*}
\end{mathbox}

Other parameters
\begin{myitem}
	\item[] - $\Xi$: positive diagonal random matrix with
	$\kappa_2(\Xi)\approx 10^{3}$
	\item[] - $\mu=0.001$
	\item[] - stopping criterion:
	$
	\dfrac{\| c - Au\|_2}{\|c\|_2}<10^{-7}
	$\medskip
	\item[] -  Initial guess: zero vector
	\item[] -  maximum iteration steps: $1000$
\end{myitem}

\footnote{[BN06]
	M. Benzi and M. K. Ng,
	\emph{Preconditioned iterative methods for weighted toeplitz
		least squares problems},
	SIAM J. Matrix Anal. Appl., 27 (2006), 1106--1124. }
\end{frame}

\begin{frame}
\frametitle{Examples: One-Dimensional Problem \mycite{BN06}}

\begin{Boxblue}
	$K=[t_{ij}]\in\R^{n\times n}$ is a Toeplitz matrix defined by
	\begin{align*}
	\text{(i)}\quad
	& t_{ij}=\frac{1}{\sqrt{|i-j|}+1}
	&\to&\ \textcolor{blue}{\textit{well-conditioned}} \\
	\text{(ii)}\quad
	& t_{ij}=\dfrac{1}{\sqrt{2\pi}\sigma}e^{\frac{-|i-j|^2}{2\sigma^2}}
	\ \text{with}\ \sigma=2
	&\to&\ \textcolor{blue}{ill-conditioned}
	\end{align*}
\end{Boxblue}

Other parameters
\footnote{[BN06]
	M. Benzi and M. K. Ng,
	\emph{Preconditioned iterative methods for weighted toeplitz
		least squares problems},
	SIAM J. Matrix Anal. Appl., 27 (2006), 1106--1124. }
\end{frame}

\begin{frame}
\frametitle{Examples: One-Dimensional Problem \mycite{BN06}}

\begin{Boxred}
$K=[t_{ij}]\in\R^{n\times n}$ is a Toeplitz matrix defined by
\begin{align*}
\text{(i)}\quad
& t_{ij}=\frac{1}{\sqrt{|i-j|}+1}
&\to&\ \textcolor{blue}{\textit{well-conditioned}} \\
\text{(ii)}\quad
& t_{ij}=\dfrac{1}{\sqrt{2\pi}\sigma}e^{\frac{-|i-j|^2}{2\sigma^2}}
\ \text{with}\ \sigma=2
&\to&\ \textcolor{blue}{ill-conditioned}
\end{align*}
\end{Boxred}

Other parameters
\footnote{[BN06]
	M. Benzi and M. K. Ng,
	\emph{Preconditioned iterative methods for weighted toeplitz
		least squares problems},
	SIAM J. Matrix Anal. Appl., 27 (2006), 1106--1124. }
\end{frame}

\begin{frame}
\frametitle{Equivalent Linear Systems -- Augmented System}
\begin{myitem}
	\item Augmented system associated with WTLS problem
	$$
	\bbm W & K \\ K^T & -\mu I \ebm \bbm y\\x\ebm = \bbm f\\ 0\ebm,
	$$
	where $W = (\Xi^T \Xi)^{-1}$ and $y = \Xi^T \Xi(f - Kx)$
	
	\item[] \zhuu\ This is a generalized saddle point problem
	
	\item[] \zhuu\ Many solution methods are available
	\begin{myitem}
		\item[] - Uzawa, HSS, GSOR, ...
		\item[] - preconditioned Krylov subspace methods
	\end{myitem}
	\medskip
	
	\qquad
	\begin{Boxred}[0.7\textwidth]
		\textcolor{blue}{How to find a good preconditioner?}
	\end{Boxred}
	
	\qquad
	\begin{Boxblue}[0.7\textwidth]
		\textcolor{blue}{How to find a good preconditioner?}
	\end{Boxblue}
	
\end{myitem}
\end{frame}

\begin{frame}
\frametitle{顶顶顶顶HSS  \mycite{BN06}}

Rewrite the augmented system into nonsymmetrix form
$$
\bbm W & K \\ -K^T & \mu I \ebm \bbm y\\x\ebm = \bbm f\\ 0\ebm
\quad\text{or}\quad
Au=c
$$

\begin{myitem}
	\item Hermitian and skew-Hermitian splitting \mycite{BGN03}
	$$ A=H+S $$
	where
	$$
	H = \bbm W & 0 & A\\ 0 & \mu I\\  W & 0\\ \ebm \ \text{and}\
	S = \bbm 0 & K\\ -K^T & 0  0 & K & a_{12}^{cd}\\ \ebm
	$$
	
\end{myitem}

\footnote{[BN06]
	M. Benzi and M. K. Ng,
	\emph{Preconditioned iterative methods for weighted toeplitz
		least squares problems},
	SIAM J. Matrix Anal. Appl., 27 (2006), 1106--1124. }
\footnote{[BGN03]
	Z.-Z. Bai, G. H. Golub and M. K. Ng,
	\emph{Hermitian and skew-Hermitian splitting methods for
		non-Hermitian positive definite linear systems},
	SIAM J. Matrix Anal. Appl., 24 (2003), 603--626. }

\end{frame}

\section{数值实验}

\begin{frame}{数值算例}
	\begin{example}
		这是第一个例子。
	\end{example}
\end{frame}

\begin{frame}{数值算例(续)}

\begin{center}
	{Numerical results for Example 1}\smallskip

	\begin{tabular}{ccccccccccc} \toprule
   	 	&&&\multicolumn{2}{c}{GMRES(C)}
    	&&\multicolumn{2}{c}{GMRES(L)}
    	&&\multicolumn{2}{c}{GMRES(P)} \\ \cmidrule{4-5}\cmidrule{7-8}\cmidrule{10-11}
	 	$\theta$ & $N$ && Iter & CPU && Iter & CPU && Iter & CPU \\\hline
 	 	0.5
	 	& $2^{11}$ &&  33&    0.04 &&  13&    0.02 &&  12&   0.01 \\
	 	& $2^{12}$ &&  33&    0.12 &&  14&    0.04 &&  12&   0.03 \\
	 	& $2^{13}$ &&  33&    0.26 &&  14&    0.09 &&  12&   0.08 \\
     	& $2^{14}$ &&  33&    0.53 &&  15&    0.19 &&  12&   0.15 \\ \midrule
 	 	0.8                                         
	 	& $2^{11}$ &&  33&    0.04 &&  13&    0.02 &&  12&   0.01 \\
	 	& $2^{12}$ &&  33&    0.11 &&  14&    0.04 &&  12&   0.03 \\
     	& $2^{13}$ &&  33&    0.25 &&  15&    0.10 &&  12&   0.08 \\
     	& $2^{14}$ &&  33&    0.53 &&  16&    0.21 &&  12&   0.15 \\ \bottomrule
	\end{tabular}
\end{center}

\end{frame}


\newcolumntype{A}{>{\columncolor{yellow!10!white}}c}
\newcolumntype{B}{>{\columncolor{blue!5!white}}c}
\newcolumntype{C}{>{\columncolor{green!10!white}}c}


\begin{frame}{数值算例 (续)}

\begin{center}
	{Numerical results for $\tcbhighmathinner{\beta=0.1}$}\smallskip
	
	\begin{tabular}{AABBCCBB} \toprule
		&&\multicolumn{2}{B}{GMRES(C)}
		&\multicolumn{2}{C}{GMRES(L)}
		&\multicolumn{2}{B}{GMRES(P)} \\ \cline{3-8}
		$\theta$ & $N$ & Iter & CPU & Iter & CPU & Iter & CPU \\\hline
		0.5
		& $2^{11}$ &  33&    0.04 &  13&    0.02 &  12&   0.01 \\
		& $2^{12}$ &  33&    0.12 &  14&    0.04 &  12&   0.03 \\
		& $2^{13}$ &  33&    0.26 &  14&    0.09 &  12&   0.08 \\
		& $2^{14}$ &  33&    0.53 &  15&    0.19 &  12&   0.15 \\ \hline
		0.8
		& $2^{11}$ &  33&    0.04 &  13&    0.02 &  12&   0.01 \\
		& $2^{12}$ &  33&    0.11 &  14&    0.04 &  12&   0.03 \\
		& $2^{13}$ &  33&    0.25 &  15&    0.10 &  12&   0.08 \\
		& $2^{14}$ &  33&    0.53 &  16&    0.21 &  12&   0.15 \\ \bottomrule
	\end{tabular}
\end{center}

\end{frame}


\begin{frame}{表格}
如表11所示\ref{tab11}
\begin{center}
	\begin{table}
		\caption{Prediction results of MLP and LSTM  on the dataset.}
		\centering{}%
		\footnotesize{}
		\begin{tabular}{p{1.0cm}<{\centering} p{1cm}<{\centering} p{2cm}<{\centering} p{1.5cm}<{\centering} p{1.5cm}<{\centering} p{1.5cm}<{\centering}}
			%\hline
			
			\toprule 
			&  & nondisruptive pulse  & \multicolumn{3}{c}{  disruptive pulse  }\\
			\cmidrule{4-6}
			%\cline{4-6}
			& indicators & $f_{nd}(\%)$ & $s_{d}(\%)$ & $l_{d}(\%)$ & $f_{d}(\%)$ \\
			%\hline
			\midrule 
			\multirow{4}{*}{MLP} & train set & 12/224 (5.35) & 223/327 (68.19) & 8/327 (2.44) & 96/327 (29.35)\\
			\\
			& validation & 8/224 (3.57) & 225/327 (68.81) & 11/327 (3.36) & 91/327 (27.83)\\
			\\
			& test & 8/224(3.57) & 220/326 (67.48) & 4/326 (1.23) & 102/326 (31.29)\\
			\\
			& total & 28/672 (4.16) & 668/980 (68.16) & 23/980 (2.35) & 289/980 (29.49) \\
			\\
			\\
			\multirow{4}{*}{LSTM} & training & 9/224 (4.02) & 243/327 (74.31) & 1/327 (0.30) & 83/327 (25.38)\\
			\\
			& validation  & 7/224 (3.12)  & 252/327 (77.06) & 3/327 (0.92) & 72/327 (22.02)\\
			\\
			& test & 5/224 (2.23) & 239/326(73.31) & 3/326 (0.92) & 84/326 (25.77)\\
			\\
			& total & 21/672 (3.12) & 734/980 (74.89) & 7/980 (0.71) & 239/980 (24.39)\\
			
			\bottomrule
			%\hline
		\end{tabular}
		\label{tab11}
	\end{table}
\end{center}
\end{frame}

\begin{frame}{图片}
\begin{figure}[htbp]
	\centering
	\subfigure[第67118炮的预测结果]{
		\includegraphics[scale=0.2]{./67039.eps}}
	\subfigure[第70055炮的预测结果]{
		\includegraphics[scale=0.2]{./67039.eps}}
	\caption{同时被LSTM和MLP错误预测的第67118炮和第70055炮非破裂炮。(共21炮)}
	\label{fig9}
\end{figure}
\end{frame}


\section{结论与展望}

\begin{frame}{结论与展望}
	\begin{figure}[htbp]
	\centering
	\includegraphics[scale=0.1]{ustc_logo_fig_new}
	\caption{Impact factor of diagnostics under four indicates by `leave one out ' method}
	\label{fig1}
	\end{figure}
  这里是结论与展望 conclusion 和 remarks

\end{frame}


%块环境: block, exampleblock, alertblock → 与定理环境类似, 可自选标题

%多栏显示:在我们做讲稿的时候,有时候为了使得幻灯片更加工整,充实,会在一旁插入图片、表格或者说明性的文字,这个时候可以使用 beamer 中的多栏环境(columns)或者 L A TEX 中的子页环境(minipage)。
%多栏的对齐方式- t,c,b,T 

\begin{frame}{实验结果}
\begin{columns}[T]
	\begin{column}{0.30\textwidth}
		\epsfig{figure=67039.eps,width=\textwidth}
		\epsfig{figure=67039.eps,width=\textwidth}
		八号或或或或或或或
	\end{column}
	\begin{column}{0.30\textwidth}
		\begin{Block}{结论1}
			\begin{itemize}
				\item 实验结果如左图所示
				\item 实验结果如左图所示
				\item 实验结果如左图所示
				\item 实验结果如左图所示
				\item 实验结果如左图所示
				\item 实验结果如左图所示
			\end{itemize}
		\end{Block}
	\end{column}
	
	\begin{column}{0.30\textwidth}
		\begin{block}{结论2}
			\begin{itemize}
				\item 实验结果如左图所示
				\item 实验结果如左图所示
				\item 实验结果如左图所示
				\item 实验结果如左图所示
				\item 实验结果如左图所示
				\item 实验结果如左图所示
			\end{itemize}
		\end{block}
	\end{column}
\end{columns}
\end{frame}


%block, exampleblock, alertblock,Block

\begin{frame}{实验结果(续1)}
\begin{columns}[T,onlytextwidth]
	
	\begin{column}{0.30\textwidth}
		\begin{block}{结论6}
			\begin{itemize}
				\item 实验结果如左图所示
				\item 实验结果如左图所示
				\item 实验结果如左图所示
				\item 实验结果如左图所示
				\item 实验结果如左图所示
				\item 实验结果如左图所示
			\end{itemize}
		\end{block}
	\end{column}
	
	\begin{column}{0.30\textwidth}
		\begin{block}{结论7}
			\begin{description}
				\item[11] 实验结果如左图所示
				\item[2] 实验结果如左图所示
				\item[3] 实验结果如左图所示
				
			\end{description}
		\end{block}
	\end{column}
	
	\begin{column}{0.30\textwidth}
		\begin{block}{结论8}
			\begin{enumerate}
				\item 实验结果如左图所示
				\item 实验结果如左图所示
				\item 实验结果如左图所示
				\item 实验结果如左图所示
				\item 实验结果如左图所示
				\item 实验结果如左图所示
			\end{enumerate}
		\end{block}
	\end{column}
\end{columns}

\alert{这是由 \\ alert 输出的....}

\end{frame}

\begin{frame}{实验结果(续2)}
\begin{columns}[T,onlytextwidth]
	
	\begin{column}{0.30\textwidth}
		\begin{alertblock}{结论10}
			\begin{itemize}
				\item 实验结果如左图所示
				\item 实验结果如左图所示
				\item 实验结果如左图所示
				\item 实验结果如左图所示
				\item 实验结果如左图所示
				\item 实验结果如左图所示
			\end{itemize}
		\end{alertblock}
	\end{column}
	
	\begin{column}{0.30\textwidth}
		\begin{alertblock}{结论11}
			\begin{description}
				\item[11] 实验结果如左图所示
				\item[2] 实验结果如左图所示
				\item[3] 实验结果如左图所示
				
			\end{description}
		\end{alertblock}
	\end{column}
	
	\begin{column}{0.30\textwidth}
		\begin{alertblock}{结论12}
			\begin{enumerate}
				\item 实验结果如左图所示
				\item 实验结果如左图所示
				\item 实验结果如左图所示
				\item 实验结果如左图所示
				\item 实验结果如左图所示
				\item 实验结果如左图所示
			\end{enumerate}
		\end{alertblock}
	\end{column}
\end{columns}

\alert{这是由 \\ alert 输出的....}

\end{frame}

\begin{frame}{实验结果(续3)}
\begin{columns}[T,onlytextwidth]
	
	\begin{column}{0.30\textwidth}
		\begin{exampleblock}{结论13}
			\begin{itemize}
				\item 实验结果如左图所示
				\item 实验结果如左图所示
				\item 实验结果如左图所示
				\item 实验结果如左图所示
				\item 实验结果如左图所示
				\item 实验结果如左图所示
			\end{itemize}
		\end{exampleblock}
	\end{column}
	
	\begin{column}{0.30\textwidth}
		\begin{exampleblock}{结论14}
			\begin{description}
				\item[11] 实验结果如左图所示
				\item[2] 实验结果如左图所示
				\item[3] 实验结果如左图所示
				
			\end{description}
		\end{exampleblock}
	\end{column}
	
	\begin{column}{0.30\textwidth}
		\begin{exampleblock}{结论15}
			\begin{enumerate}
				\item 实验结果如左图所示
				\item 实验结果如左图所示
				\item 实验结果如左图所示
				\item 实验结果如左图所示
				\item 实验结果如左图所示
				\item 实验结果如左图所示
			\end{enumerate}
		\end{exampleblock}
	\end{column}
\end{columns}

\alert{这是由 \\ alert 输出的....}

\end{frame}

\begin{frame}{实验结果(续4)}
\begin{columns}[T,onlytextwidth]
	
	\begin{column}{0.30\textwidth}
		\begin{Block}{结论13}
			\begin{itemize}
				\item 实验结果如左图所示
				\item 实验结果如左图所示
				\item 实验结果如左图所示
				\item 实验结果如左图所示
				\item 实验结果如左图所示
				\item 实验结果如左图所示
			\end{itemize}
		\end{Block}
	\end{column}
	
	\begin{column}{0.30\textwidth}
		\begin{Block}{结论14}
			\begin{description}
				\item[11] 实验结果如左图所示
				\item[2] 实验结果如左图所示
				\item[3] 实验结果如左图所示
				
			\end{description}
		\end{Block}
	\end{column}
	
	\begin{column}{0.30\textwidth}
		\begin{Block}{结论15}
			\begin{enumerate}
				\item 实验结果如左图所示
				\item 实验结果如左图所示
				\item 实验结果如左图所示
				\item 实验结果如左图所示
				\item 实验结果如左图所示
				\item 实验结果如左图所示
			\end{enumerate}
		\end{Block}
	\end{column}
\end{columns}

\alert{这是由 \\ alert 输出的....}

\end{frame}

%三类列表环境,包括无序列表(itemize)、有序列表(enumerate)、描述列表(description)。其中前两者非常常用,使用非常简单

\begin{frame}{无序列表(itemize)}
	\begin{itemize}
		\item 实验结果如左图所示
		\item 实验结果如左图所示
		\item 实验结果如左图所示
		\item 实验结果如左图所示
		\item 实验结果如左图所示
		\item 实验结果如左图所示
	\end{itemize}
\end{frame}

\begin{frame}{有序列表(enumerate)}
	\begin{enumerate}
	\item 实验结果如左图所示
	\item 实验结果如左图所示
	\item 实验结果如左图所示
	\item 实验结果如左图所示
	\item 实验结果如左图所示
	\item 实验结果如左图所示
	\end{enumerate}
\end{frame}

\begin{frame}{描述列表(description)}
	\begin{description}
	\item[First Item] 实验结果如左图所示
	\item[First Item] 实验结果如左图所示
	\item[First Item] 实验结果如左图所示
	\item[First Item] 实验结果如左图所示
	\item[First Item] 实验结果如左图所示
	\item[First Item] 实验结果如左图所示
	\end{description}
\end{frame}

\begin{frame}[c,plain]
\begin{center}
\Huge\color{red}\heiti\bfseries 谢\quad 谢!

  Thanks!
\end{center}
\end{frame}

\end{document} 
