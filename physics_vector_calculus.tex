%\documentclass[a4paper]{article}
%\usepackage{xeCJK} %调用 xeCJK 宏包
%\setCJKmainfont{SimSun} %设置 CJK 主字体为 SimSun (宋体)
\documentclass[UTF8]{ctexart}

\usepackage{fontspec}   %修改字体
\setCJKmainfont{SimSun}
%\setmainfont{Times New Roman}
\usepackage{authblk}
\usepackage{relsize}
\usepackage{amsmath}
\usepackage{amssymb}
\usepackage{amsthm}
\usepackage{cases}
\usepackage{empheq}
\usepackage{latexsym}
%\usepackage{CJKpunct}
\usepackage{graphicx} %插入图片的宏包
\usepackage{float} %设置图片浮动位置的宏包
\usepackage{stfloats} %设置图片浮动位置的宏包
\usepackage{subfigure} %插入多图时用子图显示的宏包
\usepackage[marginal]{footmisc}
\renewcommand{\thefootnote}{}
\usepackage{caption}

%设置纸张大小间距
\usepackage{geometry}
%\geometry{papersize={20cm,15cm}}%设置纸张的长宽
\geometry{a4paper,left = 1.6cm,right = 1.6cm,top = 3cm,bottom = 3cm}%设置页边距

\usepackage{multicol}%分栏宏
\usepackage{supertabular}
%
\usepackage{setspace}

\usepackage[colorlinks = true]{hyperref}

%设置页眉页脚
\usepackage{fancyhdr}
\pagestyle{fancy}
\lhead{\author}
\chead{核聚变与等离子体物理}
\rhead{\thepage}
%\lfoot{}
%\cfoot{}
%\rfoot{}
\renewcommand{\headrulewidth}{0.4pt}
\renewcommand{\headwidth}{\textwidth}
\renewcommand{\footrulewidth}{0.2pt}

%首行缩进
\usepackage{indentfirst}
\setlength{\parindent}{2em}
%然后在需要缩进的文段前添加



\renewcommand\Authands{,}
\usepackage{amsmath,amssymb,wasysym,enumerate,xcolor}

%%%%%%%%%% Start TeXmacs macros
\newcommand{\tmcolor}[2]{{\color{#1}{#2}}}
\newcommand{\tmmathbf}[1]{\ensuremath{\boldsymbol{#1}}}
\newcommand{\tmop}[1]{\ensuremath{\operatorname{#1}}}
\newcommand{\tmtextbf}[1]{{\bfseries{#1}}}
\newenvironment{enumeratealphacap}{\begin{enumerate}[A.] }{\end{enumerate}}
\newenvironment{enumeratenumeric}{\begin{enumerate}[1.] }{\end{enumerate}}
\newenvironment{enumerateroman}{\begin{enumerate}[i.] }{\end{enumerate}}
%%%%%%%%%% End TeXmacs macros

\begin{document}

\title{物理中的矢量运算总结}

\author{陈俊杰}

\date{2020,7,4}

\maketitle

\begin{abstract}
  文中加粗的为矢量,正常字体为标量。
\end{abstract}

\section{Latex公式样例、符号大全}




{\Large 行内公式:}

LaTeX进入行内公式最为简单的方式是输入用一个\$包裹的表达式。其实,用  $\backslash$(和$\backslash$) 包裹也会有一样的效果。但其实它们两个都是一种“环境”的简便表达。这个环境就是 math。总的而言,下面这三个表达方式是等价的。

$\frac{x}{y}$

$\sin{x}$

\( \cos{y} \)

\begin{math}
\tan{z}
\end{math}


{\Large 单行公式:}

最为基础的行间公式的输出方式有三种,分别是用\$\$、 $\backslash$[和$\backslash$]以及 displaymath环境。

这是一个例子:


$$\lim\limits_{r\rightarrow{0^{+}}}{\frac{1}{r^{3}}\iiint\limits_{\Omega_{r}}f
(x,y,z)dxdydz =\frac{4}{3}\pi f(P)}
$$

$$\lim\nolimits_{r\rightarrow{0^{+}}}{\frac{1}{r^{3}}\iiint\nolimits_{\Omega_{r}}f
	(x,y,z)dxdydz =\frac{4}{3}\pi f(P)}
$$

\[
\forall \epsilon >0, \exists \delta >0, \forall Q \in\Omega_{\delta},\vert{f(Q)-f(P)}\vert<\epsilon
\]



\begin{displaymath}
\frac{4}{3}\pi \delta^{3} (f(P)-\epsilon) < \iiint\limits_{\Omega_{\delta}} f(x,y,z)dxdydz < \frac{4}{3}\pi \delta^{3}(f(P)+\epsilon)
\end{displaymath}


其中,第一种方式不推荐使用。这是plain TeX时代就存在的命令,使用的是固定行距,不利于文章的后期调整。一般而言使用较多的是第二种方法,即用 $\backslash$[和 $\backslash$]来进入行间的数学模式。

当然,除了这种最为基础的行间公式,也存在较为高级的环境。

我们可以使用 amsmath提供的 equation环境来达到为自己的公式编号的目的。这样在交叉引用以及后期编辑的时候都有更大的优势。

{\Large 多行公式:}

有些时候我们需要输入的公式特别长,导致我们不得不手动为他们换行;或者某几个公式是一组,我们需要将他们放在一起;还有些时候我们需要输入分段函数,给公式加上一个在左边的大括号。这些需求都可以通过 amsmath宏包提供的丰富环境达成。(后文中的大量环境都需要 amsmath宏包,大家自己尝试时一定不要忘记使用!)

对于那些特别长的但是不需要对齐的公式,我们可以使用 multiline环境,同时需要注意,这个环境也是默认有编号的。而对于那些需要对齐的公式,我们就需要使用 aligned次环境来达到目的。使用 alinged次环境的时候需要先进入化学环境;另外,在 aligned环境中有着一个特殊的符号 \&,这个符号是用于对齐的,LaTeX会自动地以 \&为标准对齐两边的内容。在 multiline和 aligned这两个环境中,符号 $\backslash$$\backslash$是用来换行的。以下是一个例子。

\begin{multline}
x =a+b+c + \\
d+e+f
\end{multline}

\begin{multline}
U3 = Q + W \\
pV = nRT
\end{multline}


使用 aligned / gathered 环境,不须加编号则使用$\backslash$[ $\backslash$]包裹。
\[
\begin{aligned}
	x =&a+b+c\\
	&d+e+f
\end{aligned}
\]

使用 aligned / gathered 环境,不须加编号则使用$\backslash$[ $\backslash$]包裹。

\[
\begin{aligned}
U4 &= Q + W \\
pV &= nRT
\end{aligned}
\]


有些时候我们需要将几个公式组成一组。这个时候我们可以使用两种环境,分别是 gather和 align。其中, gather环境是不对齐的,而 align环境是对齐的。在默认条件下它们都带有编号。如果不需要编号可在一行的最后加入$\backslash$notag命令,也可以使用 tag\{·\}来自定义改行的公式的编号。下面是一个例子。

\begin{gather}
U =Q +W \tag{1.1}\\
pV =nRT
\end{gather}

\begin{gather}
U1 = Q + W \\
pV = nRT
\end{gather}





\begin{align}
U2 &= Q + W \\
(p + \frac{a}{{V_m}^2})(V_m + b) &= RT
\end{align}

\begin{equation}
n_{\mathrm{e}}^{\mathrm{max}} \propto \frac{B_{\mathrm{T}}}{Rq_{\alpha}}
\end{equation}



\textcolor{red}{\Large  分别编号:}\\
\textcolor{blue}{\large 分别编号,需要对齐:}


使用 align 环境,在需要对齐的地方加 \&

\begin{align} 
a & = b + c \label{eq:eq1}
\\[3pt]
& = d + e  \label{eq:eq2}
\end{align}
% 分别交叉引用
式 (\ref{eq:eq1}) 和式 (\ref{eq:eq2}) 采用 align 对齐环境

\begin{subequations}
	\begin{align}
	\vec{f}_{t} &= \sigma (W_f \cdot [\vec{h}_{t-1},\vec{x}_t] + \vec{b}_f) \\
	\vec{i}_{t} &= \sigma (W_i \cdot [\vec{h}_{t-1},\vec{x}_t] + \vec{b}_i) \\
	\widetilde{\vec{c}}_t &= tanh (W_c \cdot [\vec{h}_{t-1},\vec{x}_t] + \vec{b}_c) \\
	\vec{c}_t &= \vec{f}_t \circ \vec{c}_{t-1} + \vec{i}_t \circ \widetilde{\vec{c}}_t \\
	\vec{o}_t &= \sigma (W_o \cdot [\vec{h}_{t-1},\vec{x}_t] + \vec{b}_o) \\
	\vec{h}_t &= \vec{o}_t \circ tanh(\vec{c}_t)
	\end{align}
\end{subequations}

\noindent
\textcolor{blue}{\large 使用 $\backslash$notag 对某行不编号:}
\begin{align} 
a = {}& b + c \\ 
= {}& d + e + f + g + h + i + j + k + l \notag \\ 
&{} + m + n + o \\ 
= {}& p + q + r + s 
\end{align}

注意: 因为此处在 + (二元运算符)处对齐,所以应该在对齐符号 \& 前后使用一个占位符 {} 来避免不正确的缩进.

\textcolor{blue}{\large 多列对齐,编号:}


\begin{align} 
a &=1 & b &=2 & c &=3 \\ 
d &=-1 & e &=-2 & f &=-5 
\end{align}


\textcolor{blue}{\large 多列对齐,不编号:}


\begin{align*} 
a &=1 & b &=2 & c &=3 \\ 
d &=-1 & e &=-2 & f &=-5 
\end{align*}

\textcolor{blue}{\large 不需要对齐:}

使用 gather 环境:

\begin{gather} 
a = b + c \\ 
d = e + f + g \\ 
h + i = j + k \notag \\ 
l + m = n 
\end{gather}

使用 gather* 环境:

\begin{gather*} 
a = b + c \\ 
d = e + f + g \\ 
h + i = j + k \notag \\ 
l + m = n 
\end{gather*}


使用 align* 环境,不对齐:

\begin{align*} 
a = b + c \\ 
d = e + f + g \\ 
h + i = j + k \notag \\ 
l + m = n 
\end{align*}

使用 align* 和\& 环境,对齐:
\begin{align*} 
a &= b + c \\ 
d &= e + f + g \\ 
h + i &= j + k \notag \\ 
l + m &= n 
\end{align*}

\textcolor{purple}{注意: 若不需要编号则使用 align* 或 gather*}

\textcolor{red}{\Large  统一编号:}\\

使用 aligned / gathered 环境,并且依赖 $\backslash$begin\{equation\} $\backslash$end\{equation\},若不须加编号则使用$\backslash$[  $\backslash$]包裹。\\

使用 aligned 并且使用$\backslash$begin\{equation\} $\backslash$end\{equation\}包裹,并使用\&对齐。


\begin{equation}
\begin{aligned} 
a &= b + c \\
d &= e + f + g \\
h + i &= j + k \\
l + m &= n
\end{aligned}
\end{equation}


使用 gathered 并且使用$\backslash$begin\{equation\} $\backslash$end\{equation\}包裹,gathered无法对齐。

\begin{equation}
\begin{gathered} 
a = b + c \\
d = e + f + g \\
h + i = j + k \\
l + m = n
\end{gathered}
\end{equation}

使用gathered并且使用$\backslash$[  $\backslash$]包裹,不需要编号。

\[
\begin{gathered}
a = b + c \\
d = e + f + g \\
h + i = j + k \\
l + m = n
\end{gathered}
\]

使用 aligned 并且使用$\backslash$[  $\backslash$]包裹,不需要编号,不对齐。

\[
\begin{aligned}
a = b + c \\
d = e + f + g \\
h + i = j + k \\
l + m = n
\end{aligned}
\]


使用 aligned 并且使用$\backslash$[  $\backslash$]包裹,不需要编号,对齐。

\[
\begin{aligned}
a &= b + c \\
d &= e + f + g \\
h + i &= j + k \\
l + m &= n
\end{aligned}
\]


\textcolor{red}{\Large  添加定界符:}\\

有时需要在多行公式中添加定界符,最常用的就是在公式左端加上花括号 \{,下面介绍 3 种方法实现该功能。
	
实际上,除了 cases 环境,其他方法都可以实现其他定界符,尤其是最后一种。

\textcolor{blue}{\large $\backslash$left\{ $\backslash$right :}


不对齐带编号:
\begin{equation}
\left\{
\begin{gathered}
a_{11} x_{1} + a_{12} x_{2} + a_{13} x_{2} = b_{1}
\\[3pt]
a_{21} x_{1} + a_{22} x_{3} + a_{23} x_{3} = b_{2}
\end{gathered}
\right.
\end{equation}

对齐不带编号:
\[
\left\{
\begin{aligned}
a_{11} x_{1} + a_{12} x_{2} + a_{13} x_{2} = b_{1}
\\[3pt]
a_{22} x_{3} + a_{23} x_{3} = b_{2}
\end{aligned}
\right.
\]

\textcolor{blue}{\large array 环境:}
\[
|x| = \left\{
\begin{array}{rl}
-x & \mbox{if } x < 0,\\ 
0 & \mbox{if } x = 0,\\ 
x & \mbox{if } x > 0. 
\end{array} \right.
\]

\textcolor{blue}{\large cases 环境:}
分别编号:\\

需要在导言区载入 cases 宏包 $\backslash$usepackage\{cases\},并且放在 amsmath 之后.

\begin{numcases} {|x| =}
	-x & \mbox{if } x < 0 \label{eq22},\\
	0 & \mbox{if } x = 0,\\
	x & \mbox{if } x > 0.
\end{numcases}

\begin{subnumcases} {\label{eq24a} |x| =}
	-x & \mbox{if } x < 0,\\
	0 & \mbox{if } x = 0\label{eq24b},\\
	x & \mbox{if } x > 0.
\end{subnumcases}

式 (\ref{eq22}) 和 式 (\ref{eq24a}) 和 式 (\ref{eq24b})



统一编号:\\

\begin{equation} |x| =
\begin{cases}
-x & \mbox{if } x < 0,\\
0 & \mbox{if } x = 0,\\
x & \mbox{if } x > 0.
\end{cases}
\end{equation}

\begin{equation}
f(x) = 
\begin{cases}
-x + 1, \quad x \leq 1\\
x - 1, \quad x > 1
\end{cases}
\end{equation}

\begin{equation}
f(x) = \left\{
\begin{aligned}
x, x > 0 \\
\frac{x}{a}, x \leqslant 0 \\
\end{aligned}
\right.
\end{equation}

注意与 array 环境的区别.


无编号:\\
如果大家需要输入一个分段的函数,下面是一个例子。

\[ f(x)=
\begin{cases}
-x +1
,\quad x \leq 1\\
x -1,\quad x >1
\end{cases}
\]


*********************************************** 分割线 ********************************************


	\[
	\iiint\limits_{\Omega_{r}}f(x,y,z)dxdydz
	\]

\begin{equation} \label{eq1}
	E = mc^2
\end{equation}


多行公式--带编号
\begin{gather}
a + b +c = b + a \\
1+2 = 2 + 1
\end{gather}

多行公式--不带编号1
\begin{gather*}
a + b = b + a \\
1+2 = 2 + 1
\end{gather*}

多行公式--带编号2 $\backslash$notag 阻止编号
\begin{gather}
a + b = b + a \notag \\
1+2 = 2 + 1 \notag
\end{gather}

按\&号对齐,--带编号
\begin{align}
a+b &= b+a \\
1+2 &=  2+1
\end{align}

按\&号对齐,--不带编号
\begin{align*}
a+b &= b+a \\
1+2 &=2+1
\end{align*}

一个公式的多行排版--带编号
\begin{equation}
\begin{split}
\cos 2x &= \cos^2 x - \sin^2x + \cos^2 x - \sin^2x + \\
&+  \cos^2 x - \sin^2x + \cos^2 x - \sin^2x + \cos^2 x - \\
&+  \sin^2x + \cos^2 x - \sin^2x + \cos^2 x - \sin^2x + \\
&+  \cos^2 x - \sin^2x + \cos^2 x - \sin^2x + \\
&=2\cos^2x-1
\end{split} 
\end{equation}

一个公式的多行排版--不带编号
\begin{equation*}
\begin{split}
\cos 2x &= \cos^2 x - \sin^2x \\
&=2\cos^2x-1
\end{split} 
\end{equation*}

case环境, text{}在数学模式中处理中文-带编号
\begin{equation}
D(x)=\begin{cases}
1, & \text{如果} x \in \mathbb{Q};\\
0, & \text{如果} x \in \mathbb{R}\setminus\mathbb{Q}
\end{cases}
\end{equation}

case环境, text{}在数学模式中处理中文-不带编号
\begin{equation*}
D(x)=\begin{cases}
1, & \text{如果} x \in \mathbb{Q};\\
0, & \text{如果} x \in \mathbb{R}\setminus\mathbb{Q}
\end{cases}
\end{equation*}

长公式「被迫」折行
使用 multline 环境,实现首行局左,中间居中,末行局右,注意: 若不需要编号则使用 multline*
\begin{multline}
p = 3x^6 + 14x^5y + 590x^4y^2 + 19x^3y^3\\ 
+ \sin{x} + \cos{y} + \tan{a} + e^{x+y} \\
- 12x^2y^4 - 12xy^5 + 2y^6 - a^3b^3
\end{multline}





\subsection{数学符号的输入}
首先,为了取得输入数学公式时的最佳体验,建议大家在导言区加载 amsmath宏包。正如上一篇推送所说,只要在导言区输入 $\backslash$usepackage{amsmath}就可以了。


\subsubsection{向量}
向量(Vectors)通常用上方有小箭头(arrow symbols)的变量表示。这可由$\backslash$vec 得到。另两个命令$\backslash$overrightarrow 和$\backslash$overleftarrow在定义从A 到B 的向量时非常有用。


\begin{displaymath}
	\vec a  \quad  \overrightarrow{AB}
\end{displaymath}


\subsubsection{分数}
输入分数的基础命令是 $\backslash$frac{·}{·},其中前面的花括号内是分子,后面的花括号内是分母。在行间公式和行内公式中, $\backslash$frac命令的输出效果是有不同的。如果想要客制化输出效果,可以用 $\backslash$dfrac命令强制输出行间公式下的分式样式;反之,可以用 $\backslash$tfrac命令强制输出行内公式的分式样式。下面是一个例子。

$$ n=\frac{p}{kT} $$

\[
\lambda =\frac{kT}{\sqrt{2}\pi d^2p}
\]

$$ n=\dfrac{p}{kT} $$

\[
\lambda =\tfrac{kT}{\sqrt{2}\pi d^2p}
\]


\subsubsection{根号和上下标}

上标是通过符号 \^来生成的;下标是用符号 \_生成的。需要注意的是LaTeX只会默认将上下标符号后的第一个字符当作是上下标的内容,需要在上下标内输入很多内容的时候要记得用花括号将内容包裹起来。

根号是用 $\backslash$sqrt{·}来生成的,需要输入n次根式的时候,需要再输入一个用中括号包裹的参数,像这样 $\backslash$sqrt[n]{·}。方根符号的大小是由LaTeX自动调整的,也可以用 $\backslash$surd命令仅仅给出符号。

下面是一个比较综合的例子。


\[
\sqrt[n]{{a_{ij}}^{e\surd{2}}}
\]


\subsubsection{最常用的运算符}
比较常用的运算符有极限 $\backslash$lim;求和 $\backslash$sum\_{·}\^{·};连乘 $\backslash$prod\_{·}\^{·};以及各类积分 $\backslash$int\_{·}\^{·}、 $\backslash$iint、 $\backslash$iiint。它们的上下标在行间公式内默认是写在右侧以适应行高的,我们可以用 $\backslash$limits来强制不压缩上下标;反之可以用 $\backslash$nolimits来压缩上下标。下面是一个例子。

$$ 
\lim_{x \to 0}{\sin x}=0 
$$

\[
 \sum\nolimits_{i=1}^{n} i  \quad  \prod_{i=1}^n
\]

\[
\sum\limits_{i=1}^{n} i  \quad  \prod_{i=1}^n
\]

\[
\sum\nolimits_{i=1}^{n} i  \quad  \prod_{i=1}^n
\]

$$ \lim\limits_{r \to {0^{+}}}{\frac{1}{r^{3}}\iiint_{\Omega_{r}}f(x,y,z)dxdydz =\frac{4}{3}\pi f(P)}
$$

\subsubsection{在表达式上下方画线和括号}
我们可以用命令 $\backslash$overline和 $\backslash$underline在表达式的上、下方画出水平线;可以用 $\backslash$vec命令来画向量;可以用 $\backslash$overbrace{·}\^{·}和 $\backslash$underbrace{·}\_{·}在表达式的上、下方给出一水平的大括号。下面是一个例子。

\[
\overline{\alpha +\beta}=\underbrace{\vec{a}+b+...+z}_{26}
\]


\[ \overline{m +n} \qquad  \underline{m+n} \]


\[
\overbrace{1,2,3,4,5,.....}^{123}  \quad
\underbrace{\vec{a}+b+...+z} \quad
sin {60}^{\circ}   \quad
\prod_{i=1}^n x_i  \quad
\lim_{x \to 0}
\]


\[
\int_{0}^\infty{fxdx}
\]


\[
\oint_{C} x^3, dx + 4y^2, dy
\]


\[
\iint_{D}^{W} , dx,dy
\]

\[
\iiint_{E}^{V} , dx,dy,dz
\]

\[
\nabla \quad \mathrm{d}x  \quad \partial x \quad \dot x \quad \ddot {XY}
\]


\subsubsection{数学符号}




\begin{table}[H]
	\begin{center}
		\caption{重音符}
		\begin{tabular}{cccccccc}

	%\hline 
	
	$\hat{a}$ & $\backslash$hat\{a\} & $\check{a}$ & $\backslash$check\{a\}  &  $\tilde{a} $   &  $\backslash$tilde{a}  & $\acute{a}$  &  $\backslash$acute{a}   \\
	    
	 $\grave{a}$   &  $\backslash$grave\{a\}  &  $\dot{a}$  & $\backslash$dot\{a\}   &  $\ddot{a}$  &  $\backslash$ddot\{a\}  &  $\breve{a}$  &  $\backslash$breve\{a\} \\
	   
	 $\bar{a}$   &  $\backslash$bar\{a\}  &  $\vec{a}$  & $\backslash$vec\{a\}   &  $\widehat{a}$  &  $\backslash$widehat\{a\}  &  $\widetilde{a}$  &  $\backslash$widetilde\{a\} \\
	 
	%\hline 
    \end{tabular}
\end{center}
\end{table}



\begin{table}[H]
	\caption{小写的希腊字母}
	\centering{}
	
	\begin{tabular}{cccccccc}
		%\hline 
		
		$\alpha$ & $\backslash$alpha & $\theta$ & $\backslash$check  &  $o$   & o  & $\upsilon$  &  $\backslash$upsilon   \\
		
		$\beta$  &  $\backslash$beta  &  $\vartheta$  & $\backslash$vartheta  &  $\pi$  & $\backslash$pi  &  $\phi$  &  $\backslash$phi \\
		
		$\gamma$ & $\backslash$gamma &  $\iota$  & $\backslash$iota   &  $\varpi$  & $\backslash$varpi &  $\varphi$  &  $\backslash$varphi\\
		
		$\delta$ & $\backslash$delta & $\kappa$ & $\backslash$kappa  &  $\rho$   & $\backslash$rho  & $\chi$  &  $\backslash$chi   \\

		$\epsilon$  &  $\backslash$epsilon  &  $\lambda$  & $\backslash$lambda  &  $\varrho$  & $\backslash$varrho  &  $\psi$  &  $\backslash$psi \\

		$\varepsilon$ & $\backslash$varepsilon &  $\mu$  & $\backslash$mu   &  $\sigma$  & $\backslash$sigma &  $\omega$  &  $\backslash$omega\\

		$\zeta$ & $\backslash$zeta &  $\nu$  & $\backslash$nu   &  $\varsigma$  & $\backslash$varsigma &  $\eta$  &  $\backslash$eta\\
		
		$\xi$ & $\backslash$xi &  $\tau$  & $\backslash$tau    \\
		%\hline 
	\end{tabular}
\end{table}





\begin{table}[H]
	\begin{center}
		\caption{大写的希腊字母}
		\begin{tabular}{cccccccc}
			
		$\Gamma$ & $\backslash$Gamma & $\Lambda$ & $\backslash$Lambda  &  $\Sigma$   & $\backslash$Sigma  & $\Psi$  &  $\backslash$Psi   \\

		$\Delta$  &  $\backslash$Delta  &  $\Xi$  & $\backslash$Xi  &  $\Upsilon$  & $\backslash$Upsilon  &  $\Omega$  &  $\backslash$Omega \\

		$\Theta$ & $\backslash$Theta &  $\Pi$  & $\backslash$Pi   &  $\Phi$  & $\backslash$Phi 

			%\hline 
		\end{tabular}
	\end{center}
\end{table}


\begin{table}[H]
	\begin{center}
		\caption{二元关系符号}
		\begin{tabular}{cccccccc}
			
			$<$ & < & > & >  &  =   & =  & $\leq$  &  $\backslash$leq 或 $\backslash$le  \\
			
			$\geq$  &  $\backslash$geq  &  $\equiv$  & $\backslash$equiv  &  $\ll$  & $\backslash$ll  &  $\gg$  &  $\backslash$gg \\
			
			$\doteq$ & $\backslash$doteq &  $\prec$  & $\backslash$prec   &  $\succ$  & $\backslash$succ &  $\sim$ & $\backslash$sim \\
			
			
			$\preceq$  &  $\backslash$preceq  &  $\succeq$  & $\backslash$succeq  &  $\simeq$  & $\backslash$semeq  &  $\subset$  &  $\backslash$subset \\
			
			$\supset$ & $\backslash$supset &  $\approx$  & $\backslash$approx   &  $\subseteq$  & $\backslash$subseteq &  $\supseteq$ & $\backslash$supseteq \\
			
			$\cong$ & $\backslash$cong &  $\sqsubset$  & $\backslash$sqsubset   &  $\sqsupset$  & $\backslash$sqsupset &  $\Join$ & $\backslash$Join \\	
			
			$\sqsubseteq$ & $\backslash$sqsubseteq &  $\sqsupseteq$  & $\backslash$sqsupseteq   &  $\bowtie$  & $\backslash$bowtie &  $\in$ & $\backslash$in \\
				
				
			$\ni$ & $\backslash$ni &  $\propto$  & $\backslash$propto   &  $\vdash$  & $\backslash$vdash &  $\dashv$ & $\backslash$dashv \\	

			$\models$ & $\backslash$models &  $\mid$  & $\backslash$mid   &  $\parallel$  & $\backslash$parallel &  $\perp$ & $\backslash$perp \\


			$\smile$ & $\backslash$smile &  $\frown$  & $\backslash$frown   &  $\asymp$  & $\backslash$asymp &  $:$ & : \\	

			$\notin$ & $\backslash$notin &  $\neq$  & $\backslash$neq \\

				
					
			%\hline 
		\end{tabular}
	\end{center}
\end{table}


\begin{table}[H]
	\begin{center}
		\caption{二元运算符}
		\begin{tabular}{cccccccc}
			
			$+$ & + & - & -  &  $\pm$   & $\backslash$pm  & $\mp$  &  $\backslash$mp  \\
			
			$\triangleleft$  &  $\backslash$triangleleft  &  $\cdot$  & $\backslash$cdot  &  $\cap$  & $\backslash$cap  &  $\ast$  &  $\backslash$ast \\
			
			$\sqcup$ & $\backslash$sqcup &  $\sqcap$  & $\backslash$sqcap   &  $\circ$  & $\backslash$circ &  $\vee$ & $\backslash$vee \\
			
			
			$\land$  &  $\backslash$land  &  $\bullet$  & $\backslash$bullet  &  $\oplus$  & $\backslash$oplus  &  $\ominus$  &  $\backslash$ominus \\
			
			$\diamond$ & $\backslash$diamond &  $\odot$  & $\backslash$odot   &  $\oslash$  & $\backslash$oslash &  $\uplus$ & $\backslash$uplus \\
			
			$\otimes$ & $\backslash$otimes &  $\bigcirc$  & $\backslash$bigcirc   &  $\amalg$  & $\backslash$amalg &  $\bigtriangleup$ & $\backslash$bigtriangleup \\	
			
			$\bigtriangledown$ & $\backslash$bigtriangledown &  $\dagger$  & $\backslash$dagger   &  $\lhd$  & $\backslash$lhd &  $\rhd$ & $\backslash$rhd \\
			
			
			$\dagger$ & $\backslash$dagger &  $\lhd$  & $\backslash$lhd   &  $\rhd$  & $\backslash$rhd &  $\ddagger$ & $\backslash$ddagger \\	
			
			$\unlhd$ & $\backslash$unlhd &  $\unrhd$  & $\backslash$unrhd   &  $\wr$  & $\backslash$wr  \\
	
			
			%\hline 
		\end{tabular}
	\end{center}
\end{table}


\begin{table}[H]
	\begin{center}
		\caption{大运算符}
		\begin{tabular}{cccccccc}
			
			
			$\sum$  &  $\backslash$sum  &  $\bigvee$  & $\backslash$bigvee  &  $\bigoplus$  & $\backslash$bigoplus  &  $\prod$  &  $\backslash$prod \\
			
			$\bigcap$ & $\backslash$bigcap &  $\bigwedge$  & $\backslash$bigwedge   &  $\bigotimes$  & $\backslash$bigotimes &  $\coprod$ & $\backslash$coprod \\
			
			
			$\bigsqcup$  &  $\backslash$bigsqcup  &  $\bigodot$  & $\backslash$bigodot  &  $\int$  & $\backslash$int  &  $\oint$  &  $\backslash$oint \\
			
			$\biguplus$  &  $\backslash$biguplus  \\
			%\hline 
		\end{tabular}
	\end{center}
\end{table}

\begin{table}[H]
	\begin{center}
		\caption{箭头}
		\begin{tabular}{cccccccc}
			
			
			$\leftarrow$  &  $\backslash$leftarrow  &  $\longleftarrow$  & $\backslash$longleftarrow  &  $\uparrow$  & $\backslash$uparrow  &  $\rightarrow$  &  $\backslash$rightarrow \\
			
			$\longrightarrow$ & $\backslash$longrightarrow &  $\downarrow$  & $\backslash$downarrow   &  $\leftrightarrow$  & $\backslash$leftrightarrow &  $\longleftrightarrow$ & $\backslash$longleftrightarrow \\
			
			
			$\updownarrow$  &  $\backslash$updownarrow  &  $\Leftarrow$  & $\backslash$Leftarrow  &  $\Longleftarrow$  & $\backslash$Longleftarrow  &  $\Uparrow$  &  $\backslash$Uparrow \\
			
			$\Rightarrow$ & $\backslash$Rightarrow &  $\Longrightarrow$  & $\backslash$Longrightarrow   &  $\Downarrow$  & $\backslash$Downarrow &  $\Leftrightarrow$ & $\backslash$Leftrightarrow \\
			
			$\Longleftrightarrow$ & $\backslash$Longleftrightarrow &  $\Updownarrow$  & $\backslash$Updownarrow   &  $\mapsto$  & $\backslash$mapsto &  $\longmapsto$ & $\backslash$longmapsto \\	

			$\nearrow$ & $\backslash$nearrow &  $\hookleftarrow$  & $\backslash$hookleftarrow   &  $\hookrightarrow$  & $\backslash$hookrightarrow &  $\searrow$ & $\backslash$searrow \\

			$\leftharpoonup$ & $\backslash$leftharpoonup &  $\rightharpoonup$  & $\backslash$rightharpoonup   &  $\swarrow$  & $\backslash$swarrow &  $\leftharpoondown$ & $\backslash$leftharpoondown \\


			$\rightharpoondown$ & $\backslash$rightharpoondown &  $\nwarrow$  & $\backslash$nwarrow   &  $\rightleftharpoons$  & $\backslash$rightleftharpoons &  $\iff$ & $\backslash$iff \\

			$\leadsto$ & $\backslash$leadsto \\

			
			%\hline 
		\end{tabular}
	\end{center}
\end{table}

\begin{table}[H]
	\begin{center}
		\caption{定界符}
		\begin{tabular}{cccccccc}
			
			
			$($  &  (  &  $)$  & )  &  $\uparrow$  & $\backslash$uparrow  &  $\Uparrow$  &  $\backslash$Uparrow \\
			
			$[$ &  [ &  $]$  &  ]   &  $\downarrow$  & $\backslash$downarrow &  $\Downarrow$ & $\backslash$Downarrow \\
			
			
			$\lbrace$  &  $\backslash$lbrace  &  $\rbrace$  & $\backslash$rbrace  &  $\updownarrow$  & $\backslash$updownarrow  &  $\Updownarrow$  &  $\backslash$Updownarrow \\
			
			$\langle$ & $\backslash$langle &  $\rangle$  & $\backslash$rangle   &  $\vert$  & $\backslash$vert &  $\Vert$ & $\backslash$Vert \\
			
			$\lfloor$ & $\backslash$lfloor &  $\rfloor$  & $\backslash$rfloor   &  $\lceil$  & $\backslash$rceil &  $/$ & / \\	
			
			
			$\lgroup$ & $\backslash$lgroup &  $\rgroup$  & $\backslash$rgroup   &  $\lmoustache$  & $\backslash$lmoustache &  $\rmoustache$ & $\backslash$rmoustache \\
			
			$\arrowvert$ & $\backslash$arrowvert &  $\Arrowvert$  & $\backslash$Arrowvert   &  $\bracevert$  & $\backslash$bracevert  \\
		
			
			%\hline 
		\end{tabular}
	\end{center}
\end{table}


\begin{table}[H]
	\begin{center}
		\caption{其他数学符号}
		\begin{tabular}{cccccccc}
			
			
			$\dots$  &  $\backslash$dots  &  $\cdot$  & $\backslash$cdot  &  $\vdots$  & $\backslash$vdots  &  $\hbar$  &  $\backslash$hbar \\
			
			$\imath$ & $\backslash$imath &  $\jmath$  & $\backslash$jmath   &  $\ell$  & $\backslash$ell &  $\Re$ & $\backslash$Re \\
			
			
			$\Im$  &  $\backslash$Im  &  $\aleph$  & $\backslash$aleph  &  $\wp$  & $\backslash$wp  &  $\forall$  &  $\backslash$forall \\
			
			$\exists$ & $\backslash$exists &  $\mho$  & $\backslash$mho   &  $\partial$  & $\backslash$partial &  $,$ & , \\
			
			$\prime$ & $\backslash$prime &  $\emptyset$  & $\backslash$emptyset   &  $\infty$  & $\backslash$infty &  $\nabla$ & $\backslash$nabla \\	

			$\triangle$ & $\backslash$triangle &  $\Box$  & $\backslash$Box   &  $\Diamond$  & $\backslash$Diamond &  $\bot$ & $\backslash$bot \\	


			$\top$ & $\backslash$top &  $\angle$  & $\backslash$angle   &  $\surd$  & $\backslash$surd &  $\diamondsuit$ & $\backslash$diamondsuit \\	

			$\heartsuit$ & $\backslash$heartsuit &  $\clubsuit$  & $\backslash$clubsuit   &  $\spadesuit$  & $\backslash$spadesuit &  $\lnot$ & $\backslash$lnot \\	
			
			$\flat$ & $\backslash$flat &  $\natural$  & $\backslash$natural   &  $\sharp$  & $\backslash$sharp  \\				
			%\hline 
		\end{tabular}
	\end{center}
\end{table}

\begin{table}[H]
	\begin{center}
		\caption{非数学符号}
		\begin{tabular}{cccccccc}
			
			
			$\dag$  &  $\backslash$dag  &  $\S$  & $\backslash$S  &  $\copyright$  & $\backslash$copyright  &  $\ddag$  &  $\backslash$ddag \\
			
			$\P$  &  $\backslash$P  &  $\pounds$  & $\backslash$pounds   \\
			%\hline 
		\end{tabular}
	\end{center}
\end{table}


\begin{table}[H]
	\begin{center}
		\caption{AMS定界符}
		\begin{tabular}{cccccccc}
			
			
			$\ulcorner$  &  $\backslash$ulcorner  &  $\urcorner$  & $\backslash$urcorner  &  $\llcorner$  & $\backslash$llcorner  &  $\lrcorner$  &  $\backslash$lrcorner \\
			
			$\lvert$ & $\backslash$lvert &  $\rvert$  & $\backslash$rvert   &  $\lVert$  & $\backslash$lVert &  $\rVert$ & $\backslash$rVert \\
			
			%\hline 
		\end{tabular}
	\end{center}
\end{table}



\begin{table}[H]
	\begin{center}
		\caption{AMS希伯来字母}
		\begin{tabular}{cccccccc}
			
			$\digamma$  &  $\backslash$digamma  &  $\varkappa$  & $\backslash$varkappa  &  $\beth$  & $\backslash$beth  &  $\daleth$  &  $\backslash$daleth \\
			
			$\gimel$ & $\backslash$gimel \\
			
			%\hline 
		\end{tabular}
	\end{center}
\end{table}

\begin{table}[H]
	\begin{center}
		\caption{AMS二元关系符}
		\begin{tabular}{cccccc}
			
			
			$\lessdot$  &  $\backslash$lessdot  &  $\gtrdot$  & $\backslash$gtrdot  &  $\doteqdot$  & $\backslash$doteqdot  \\
			
			$\leqslant$  &  $\backslash$leqslant  &  $\geqslant$  & $\backslash$geqslant  &  $\risingdotseq$  & $\backslash$risingdotseq  \\
			
			$\leqq$  &  $\backslash$leqq  &  $\geqq$  & $\backslash$geqq  &  $\eqcirc$  & $\backslash$eqcirc  \\

			$\lessdot$  &  $\backslash$lessdot  &  $\gtrdot$  & $\backslash$gtrdot  &  $\doteqdot$  & $\backslash$doteqdot  \\

			$\lll$  &  $\backslash$lll  &  $\ggg$  & $\backslash$ggg  &  $\circeq$  & $\backslash$circeq  \\

			$\lesssim$  &  $\backslash$lesssim  &  $\gtrsim$  & $\backslash$gtrsim  &  $\triangleq$  & $\backslash$triangleq  \\


			$\lessapprox$  &  $\backslash$lessapprox  &  $\gtrapprox$  & $\backslash$gtrapprox  &  $\bumpeq$  & $\backslash$bumpeq  \\

			$\lessgtr$  &  $\backslash$lessgtr  &  $\gtrless$  & $\backslash$gtrless  &  $\Bumpeq$  & $\backslash$Bumpeq  \\

			$\lesseqgtr$  &  $\backslash$lesseqgtr  &  $\gtreqless$  & $\backslash$gtreqless  &  $\thicksim$  & $\backslash$thicksim  \\


			$\lesseqqgtr$  &  $\backslash$lesseqqgtr  &  $\gtreqqless$  & $\backslash$gtreqqless  &  $\thickapprox$  & $\backslash$thickapprox  \\

			$\preccurlyeq$  &  $\backslash$preccurlyeq  &  $\succcurlyeq$  & $\backslash$succcurlyeq  &  $\approxeq$  & $\backslash$approxeq  \\

			$\curlyeqprec$  &  $\backslash$curlyeqprec  &  $\curlyeqsucc$  & $\backslash$curlyeqsucc  &  $\backsim$  & $\backslash$backsim  \\
			
			$\precsim$  &  $\backslash$precsim  &  $\succsim$  & $\backslash$succsim  &  $\backsimeq$  & $\backslash$backsimeq  \\

			$\precapprox$  &  $\backslash$precapprox  &  $\succapprox$  & $\backslash$succapprox  &  $\vDash$  & $\backslash$vDash  \\
			
			$\subseteqq$  &  $\backslash$subseteqq  &  $\supseteqq$  & $\backslash$supseteqq  &  $\Vdash$  & $\backslash$Vdash  \\

			$\Subset$  &  $\backslash$Subset  &  $\Supset$  & $\backslash$Supset  &  $\Vvdash$  & $\backslash$Vvdash  \\

			$\sqsubset$  &  $\backslash$sqsubset  &  $\sqsubset$  & $\backslash$sqsubset  &  $\backepsilon$  & $\backslash$backepsilon  \\

			$\therefore$  &  $\backslash$therefore  &  $\because$  & $\backslash$because  &  $\varpropto$  & $\backslash$varproptosim  \\

			$\shortmid$  &  $\backslash$shortmid  &  $\shortparallel$  & $\backslash$shortparallel  &  $\between$  & $\backslash$between  \\

			$\smallsmile$  &  $\backslash$smallsmile  &  $\smallfrown$  & $\backslash$smallfrown  &  $\pitchfork$  & $\backslash$pitchfork  \\

			$\vartriangleleft$  &  $\backslash$vartriangleleft  &  $\vartriangleright$  & $\backslash$vartriangleright  &  $\blacktriangleleft$  & $\backslash$blacktriangleleft  \\

			$\trianglelefteq$  &  $\backslash$trianglelefteq  &  $\trianglerighteq$  & $\backslash$trianglerighteq  &  $\blacktriangleright$  & $\backslash$blacktriangleright  \\

				
			%\hline 
		\end{tabular}
	\end{center}
\end{table}

\begin{table}[H]
	\begin{center}
		\caption{AMS箭头}
		\begin{tabular}{cccccc}
			
			$\dashleftarrow$  &  $\backslash$dashleftarrow  &  $\dashrightarrow$  & $\backslash$dashrightarrow  &  $\multimap$  & $\backslash$multimap  \\

			$\leftleftarrows$  &  $\backslash$leftleftarrows  &  $\rightrightarrows$  & $\backslash$rightrightarrows  &  $\upuparrows$  & $\backslash$upuparrows  \\

			$\leftrightarrows$  &  $\backslash$leftrightarrows  &  $\rightleftarrows$  & $\backslash$rightleftarrows  &  $\downdownarrows$  & $\backslash$downdownarrows  \\

			$\Lleftarrow$  &  $\backslash$Lleftarrow  &  $\Rrightarrow$  & $\backslash$Rrightarrow  &  $\upharpoonleft$  & $\backslash$upharpoonleft  \\

			$\twoheadleftarrow$  &  $\backslash$twoheadleftarrow  &  $\twoheadrightarrow$  & $\backslash$twoheadrightarrow  &  $\upharpoonright$  & $\backslash$upharpoonright  \\

			$\leftarrowtail$  &  $\backslash$leftarrowtail  &  $\rightarrowtail$  & $\backslash$rightarrowtail  &  $\downharpoonright$  & $\backslash$downharpoonright  \\
			
			$\Lsh$  &  $\backslash$Lsh  &  $\Rsh$  & $\backslash$Rsh  &  $\rightsquigarrow$  & $\backslash$rightsquigarrow  \\

			$\looparrowleft$  &  $\backslash$looparrowleft  &  $\looparrowright$  & $\backslash$looparrowright  &  $\leftrightsquigarrow$  & $\backslash$leftrightsquigarrow  \\

			$\curvearrowleft$  &  $\backslash$curvearrowleft  &  $\curvearrowright$  & $\backslash$curvearrowright  &     &    \\

			$\circlearrowleft$  &  $\backslash$circlearrowleft  &  $\circlearrowright$  & $\backslash$circlearrowright  &    &    \\			
			
			
			%\hline 
		\end{tabular}
	\end{center}
\end{table}


\begin{table}[H]
	\begin{center}
		\caption{AMS二元否定关系符和箭头}
		\begin{tabular}{cccccc}
			
			$\nless$  &  $\backslash$nless  &  $\ngtr$  & $\backslash$ngtr  &  $\varsubsetneqq$  & $\backslash$varsubsetneqq  \\

			$\lneq$  &  $\backslash$lneq  &  $\gneq$  & $\backslash$gneq  &  $\varsupsetneqq$  & $\backslash$varsupsetneqq  \\

			$\nleq$  &  $\backslash$nleq  &  $\ngeq$  & $\backslash$ngeq  &  $\nsubseteqq$  & $\backslash$nsubseteqq  \\

			$\nleqslant$  &  $\backslash$nleqslant  &  $\ngeqslant$  & $\backslash$ngeqslant  &  $\nsupseteqq$  & $\backslash$nsupseteqq  \\

			$\lneqq$  &  $\backslash$lneqq  &  $\gneqq$  & $\backslash$gneqq  &  $\nmid$  & $\backslash$nmid  \\

			$\lvertneqq$  &  $\backslash$lvertneqq  &  $\gvertneqq$  & $\backslash$gvertneqq  &  $\nparallel$  & $\backslash$nparallel  \\

			$\nleqq$  &  $\backslash$nleqq  &  $\ngeqq$  & $\backslash$ngeqq  &  $\nshortparallel$  & $\backslash$nshortparallel  \\

			$\lnapprox$  &  $\backslash$lnapprox  &  $\gnapprox$  & $\backslash$gnapprox  &  $\nsim$  & $\backslash$nsim  \\

			$\nprec$  &  $\backslash$nprec  &  $\nsucc$  & $\backslash$nsucc  &  $\ncong$   &  $\backslash$ncong  \\

			$\npreceq$  &  $\backslash$npreceq  &  $\nsucceq$  & $\backslash$nsucceq  &  $\nvdash$   &  $\backslash$nvdash  \\			
			
			$\precneqq$  &  $\backslash$precneqq  &  $\succneqq$  & $\backslash$succneqq  &  $\nvDash$  & $\backslash$nvDash  \\

			$\precnapprox$  &  $\backslash$precnapprox  &  $\succnapprox$  & $\backslash$succnapprox  &  $\nVDash$  & $\backslash$nVDash  \\

			$\subsetneq$  &  $\backslash$subsetneq  &  $\supsetneq$  & $\backslash$supsetneq  &  $\ntriangleleft$  & $\backslash$ntriangleleft  \\


			$\varsubsetneq$  &  $\backslash$varsubsetneq  &  $\varsupsetneq$  & $\backslash$varsupsetneq  &  $\ntriangleright$  & $\backslash$ntriangleright  \\

			$\nsubseteq$  &  $\backslash$nsubseteq  &  $\nsupseteq$  & $\backslash$nsupseteq  &  $\ntrianglelefteq$  & $\backslash$ntrianglelefteq  \\

			$\subsetneqq$  &  $\backslash$subsetneqq  &  $\supsetneqq$  & $\backslash$supsetneqq  &  $\ntrianglerighteq$  & $\backslash$ntrianglerighteq  \\

			$\nleftarrow$  &  $\backslash$nleftarrow  &  $\nrightarrow$  & $\backslash$nrightarrow  &  $\nleftrightarrow$  & $\backslash$nleftrightarrow  \\

			$\nLeftarrow$  &  $\backslash$nLeftarrow  &  $\nRightarrow$  & $\backslash$nRightarrow  &  $\nLeftrightarrow$  & $\backslash$nLeftrightarrow  \\

						%\hline 
		\end{tabular}
	\end{center}
\end{table}






\begin{table}[H]
	\begin{center}
		\caption{AMS二元运算符}
		\begin{tabular}{cccccccc}
			
			$\dotplus$  &  $\backslash$dotplus  &  $\centerdot$  & $\backslash$centerdot  &  $\intercal$  & $\backslash$intercal  \\

			$\ltimes$  &  $\backslash$ltimes  &  $\rtimes$  & $\backslash$rtimes  &  $\divideontimes$   &  $\backslash$divideontime  \\

			$\Cup$  &  $\backslash$Cup  &  $\Cap$  & $\backslash$Cap  &  $\smallsetminus$   &  $\backslash$smallsetminus  \\			

			$\veebar$  &  $\backslash$veebar  &  $\barwedge$  & $\backslash$barwedge  &  $\doublebarwedge$  & $\backslash$doublebarwedge  \\

			$\boxplus$  &  $\backslash$boxplus  &  $\boxminus$  & $\backslash$boxminus  &  $\circleddash$  & $\backslash$circleddash  \\

			$\boxtimes$  &  $\backslash$boxtimes  &  $\boxdot$  & $\backslash$boxdot  &  $\circledcirc$  & $\backslash$circledcirc  \\


			$\leftthreetimes$  &  $\backslash$leftthreetimes  &  $\rightthreetimes$  & $\backslash$rightthreetimes  &  $\circledast$  & $\backslash$circledast  \\



			%\hline 
		\end{tabular}
	\end{center}
\end{table}


\begin{table}[H]
	\begin{center}
		\caption{AMS其他符号}
		\begin{tabular}{cccccc}
			
			$\hbar$  &  $\backslash$hbar  &  $\hslash$  & $\backslash$hslash  &  $\Bbbk$  & $\backslash$Bbbk  \\

			$\square$  &  $\backslash$square  &  $\blacksquare$  & $\backslash$blacksquare  &  $\circledS$   &  $\backslash$circledS  \\

			$\vartriangle$  &  $\backslash$vartriangle  &  $\blacktriangledown$  & $\backslash$blacktriangledown  &  $\Game$   &  $\backslash$Game  \\			

			$\lozenge$  &  $\backslash$lozenge  &  $\blacklozenge$  & $\backslash$blacklozenge  &  $\bigstar$  & $\backslash$bigstar  \\

			$\angle$  &  $\backslash$angle  &  $\measuredangle$  & $\backslash$measuredangle  &  $\sphericalangle$  & $\backslash$sphericalangle  \\

			$\diagup$  &  $\backslash$diagup  &  $\diagdown$  & $\backslash$diagdown  &  $\backprime$  & $\backslash$backprime  \\


			$\nexists$  &  $\backslash$nexists  &  $\Finv$  & $\backslash$Finv  &  $\varnothing$  & $\backslash$varnothing  \\

			$\eth$  &  $\backslash$eth  &  $\mho$  & $\backslash$mho \\


			%\hline 
		\end{tabular}
	\end{center}
\end{table}

\begin{table}[H]
	\begin{center}
		\caption{数学字体说明}
		\begin{tabular}{c c  c}
		\hline 
		例子 & 命令 & 所需宏包                   \\%\tabularnewline
		\hline 	
		$\mathrm{ABCdef}$   &  $\backslash$mathrm\{ABCdef\}   &    \\
		
		$\mathit{ABCdef}$   &  $\backslash$mathit\{ABCdef\}   &    \\
		
		$\mathnormal{ABCdef}$   &  $\backslash$mathnormal\{ABCdef\}   &    \\
		
		$\mathcal{ABCdef}$   &  $\backslash$mathcal\{ABCdef\}   &    \\


		$\mathfrak{ABCdef}$   &  $\backslash$mathfrak\{ABCdef\}   &    \\
		
		$\mathbb{ABCdef}$   &  $\backslash$mathbb\{ABCdef\}   &    \\

			%\hline 
		\end{tabular}
	\end{center}
\end{table}


\subsection{微积分}
$$
\oiint{(x-y)dxdy}
$$


$$
\oiint{(x-y)dxdy}
$$


$$
\iiint{(x-y)dxdy}
$$

$$
\iiint_{x=0,y}^{x=12,y=13}{(x-y)dxdy}
$$

$$
\oint_{\Sigma}{(x-y)dxdy}
$$



\[
\displaystyle \oiint_{x=0,y=0}^{x=12,y=22}{(x-y)dxdy}
\]



\[
\displaystyle \oiint_{0}^{\infty}{(x^2+y^2+z^2) \mathrm dx \mathrm{dy} \mathit{dz}}
\]



\[
\int_{a}^{b} x^2 dx
\]

\[
\iint_V \mu(u,v) \,du\,dv
\]

$$\iiint_V \mu(u,v,w) \,du\,dv\,dw$$

$$\iiiint_V \mu(t,u,v,w) \,dt\,du\,dv\,dw$$

$$\idotsint_V \mu(u_1,\dots,u_k) \,du_1 \dots du_k$$

Sum $\sum_{n=1}^{\infty} 2^{-n} = 1$ inside text

$$\sum_{n=1}^{\infty} 2^{-n} = 1$$

Product $\prod_{i=a}^{b} f(i)$ inside text

$$\prod_{i=a}^{b} f(i)$$

Limit $\lim_{x\to\infty} f(x)$ inside text



$$\lim_{x\to\infty} f(x)$$

Integral $\int_{a}^{b} x^2 dx$ inside text

Integral $\int\nolimits_{a}^{b} x^2 dx$ inside text

Improved integral $\int\limits_{a}^{b} x^2 dx$ inside text

Sum $\sum_{n=1}^{\infty} 2^{-n} = 1$ inside text

Sum $\sum\nolimits_{n=1}^{\infty} 2^{-n} = 1$ inside text


Improved sum $\sum\limits_{n=1}^{\infty} 2^{-n} = 1$ inside text

$\int\frac{1}{2}dx - \mathlarger{ \int\frac{1}{2}dx}$


\[
V = \iiint \limits_{V} \mathrm{d} V
\]

\subsection{矩阵的排版}
\subsubsection{矩阵的括号}
无括号
\[
\begin{matrix}
0 & 1 \\
1 & 0 
\end{matrix}
\]

小括号
\[
\begin{pmatrix}
0 & 1 \\
1 & 0 
\end{pmatrix}
\]

中括号
\[
\begin{bmatrix}
0 & 1 \\
1 & 0 
\end{bmatrix}
\]

大括号
\[
\begin{Bmatrix}
0 & 1 \\
1 & 0 
\end{Bmatrix}
\]

 单竖线
\[
\begin{vmatrix}
0 & 1 \\
1 & 0 
\end{vmatrix}
\]

双竖线
\[
\begin{Vmatrix}
0 & 1 \\
1 & 0 
\end{Vmatrix}
\]

\subsubsection{矩阵的省略号}
$\backslash$dots 横向省略号,
$\backslash$vdots 竖向省略号,
$\backslash$ddots 斜向省略号.
\[
A = \begin{bmatrix}
a_{11} & \dots & a_{1n}\\
\vdots & \ddots & \vdots \\
0 & \dots & a_{nn}
\end{bmatrix}_{n \times n}
\]


$$\begin{bmatrix}
{a_{11}}&{a_{12}}&{\cdots}&{a_{1n}}\\
{a_{21}}&{a_{22}}&{\cdots}&{a_{2n}}\\
{\vdots}&{\vdots}&{\ddots}&{\vdots}\\
{a_{m1}}&{a_{m2}}&{\cdots}&{a_{mn}}\\
\end{bmatrix}
$$


$$\begin{bmatrix}
0  &  \cdots &  12\\
\vdots &  \ddots &  \vdots\\
0 &  \cdots  &  0\\

\end{bmatrix}
$$

\subsubsection{行内小矩阵}
复数可用矩阵
\begin{math}
\left(
\begin{smallmatrix}
x & y \\ -y & x
\end{smallmatrix}
\right)
\end{math}
来表示

\subsubsection{array环境}


\[
\begin{array}{c|c}
1 & 2\\
\hline
0 & 1
\end{array}
\]

\section{常用的矢量计算}

下面式子标号为1.1,1.2,1.3$\cdots$
\begin{enumerate}
  
  \item $\nabla (\varphi \psi)$=$\varphi \nabla \psi$+$\psi \nabla \varphi$
  
  \item $\nabla \cdot (\varphi \tmmathbf{f})$=($\nabla \varphi$)$\cdot
  \tmmathbf{f}$+$\varphi \nabla \cdot \tmmathbf{f}$
  
  \item $\nabla \times (\varphi \tmmathbf{f})$=($\nabla \varphi$)$\times
  \tmmathbf{f}$+$\varphi \nabla \times \tmmathbf{f}$
  
  \item ($\nabla \varphi$)$\times \tmmathbf{f}$=(\tmtextbf{$\tmmathbf{f
  \cdot}$}$\nabla$)$\varphi$
  
  \item $\nabla \times (\varphi \tmmathbf{f})$=(\tmtextbf{$\tmmathbf{f
  \cdot}$}$\nabla$)$\varphi$+$\varphi \nabla \times \tmmathbf{f}$
  
  \item $\nabla \tmmathbf{\cdot} (\tmmathbf{f \times g}) = (\nabla \times
  \tmmathbf{f}) \tmmathbf{\cdot} \tmmathbf{g} - \tmmathbf{f \cdot} (\nabla
  \times \tmmathbf{g})$
  
  \item $\nabla \times (\tmmathbf{f \times} \tmmathbf{g}) = (\tmmathbf{g
  \cdot} \nabla) \tmmathbf{f} + (\nabla \cdot \tmmathbf{g}) \tmmathbf{f} -
  (\tmmathbf{f \cdot} \nabla) \tmmathbf{g} - (\nabla \cdot \tmmathbf{f})
  \tmmathbf{g}$
  
  \item $\nabla (\tmmathbf{f \cdot} \tmmathbf{g}) = \tmmathbf{f} \times
  (\nabla \times \tmmathbf{g})$+$(\tmmathbf{f \cdot} \nabla)
  \tmmathbf{g}$+\tmtextbf{$\tmmathbf{g \times}$}($\nabla \times
  \tmmathbf{f}$)+($\tmmathbf{g \cdot} \nabla$)$\tmmathbf{f}$
  
  \item $\nabla \times (\nabla \times \tmmathbf{f})$=$\nabla (\nabla \cdot
  \tmmathbf{f}) - \nabla^2 \tmmathbf{f}$
  
  \item $\nabla \cdot \nabla \varphi \equiv \nabla^2 \varphi$=$\Delta
  \varphi$,拉普拉斯算符
  
  \item $\nabla$($f + g$)=$\nabla f$+$\nabla g$
  
  \item $\nabla \cdot (\tmmathbf{f +} \tmmathbf{g}) = \nabla \cdot
  \tmmathbf{f}$+$\nabla \cdot \tmmathbf{g}$
  
  \item $\nabla \cdot (\nabla \times \tmmathbf{f})$=0
  
  \item $\nabla \times (\nabla f)$=0
  
  \item (\tmtextbf{$\tmmathbf{u}
  \tmmathbf{\cdot}$}$\nabla$)$\tmmathbf{u}$=$\frac{1}{2} \nabla
  (\tmmathbf{u^2}) - \tmmathbf{u \times} (\nabla \times \tmmathbf{u})$
  
  \item $\nabla
  \tmmathbf{\cdot}$(\tmtextbf{B}\tmtextbf{B})=\tmtextbf{B}($\nabla
  \tmmathbf{\cdot}
  \tmmathbf{B}$)+(\tmtextbf{B$\cdot$}$\nabla$)$\tmmathbf{B}$,其中\tmtextbf{B}\tmtextbf{B}为并矢
  
  \item $\tmmathbf{A \times} (\tmmathbf{B} \times \tmmathbf{C}) = \tmmathbf{B
  (\tmmathbf{A \cdot C})} - \tmmathbf{C (\tmmathbf{A \cdot} B)}$
  
  \item \tmtextbf{A$\cdot (\tmmathbf{B \times} C)$=$\tmmathbf{C \cdot
  (\tmmathbf{A \times} B)}$=}$\tmmathbf{B \cdot (\tmmathbf{A} \times
  \tmmathbf{C})}$
\end{enumerate}


在直角坐标系下的证明:
\begin{enumeratenumeric}
  \item
  
  \item
  
  \item
  
  \item  第四式的证明:($\nabla \varphi$)$\times
  \tmmathbf{f}$=(\tmtextbf{$\tmmathbf{f \cdot}$}$\nabla$)$\varphi$
  \begin{eqnarray*}
    (\tmmathbf{f} \cdot \nabla) \varphi & = & \left[ (\tmmathbf{e_{}}_x f_x +
    \tmmathbf{e_{}}_y f_y + \tmmathbf{e_{}}_y f_z) \cdot \left( \tmmathbf{e}_x
    \tmmathbf{} \frac{\partial}{\partial x} + \tmmathbf{e_{}}_y
    \frac{\partial}{\partial y} + \tmmathbf{e_{}}_z \frac{\partial}{\partial
    z} \right) \right] \varphi\\
    & = & \left( f_x \tmmathbf{} \tmmathbf{} \frac{\partial}{\partial x} +
    f_y \frac{\partial}{\partial y} + f_z \frac{\partial}{\partial z} \right)
    \varphi\\
    & = & f_x \tmmathbf{} \tmmathbf{} \frac{\partial \varphi}{\partial x} +
    f_y \frac{\partial \varphi}{\partial y} + f_z \frac{\partial
    \varphi}{\partial z}\\
    (\nabla \varphi) \times \tmmathbf{f} & = & \left( \tmmathbf{e}_x
    \tmmathbf{} \frac{\partial \varphi}{\partial x} + \tmmathbf{e_{}}_y
    \frac{\partial \varphi}{\partial y} + \tmmathbf{e_{}}_z \frac{\partial
    \varphi}{\partial z} \right) \times (\tmmathbf{e_{}}_x f_x +
    \tmmathbf{e_{}}_y f_y + \tmmathbf{e_{}}_y f_z)\\
    & = & f_x \tmmathbf{} \tmmathbf{} \frac{\partial \varphi}{\partial x} +
    f_y \frac{\partial \varphi}{\partial y} + f_z \frac{\partial
    \varphi}{\partial z}
  \end{eqnarray*}
  得证
  
  \item 由3,4两式即可得5式
  
  \item
  
  \item
  
  \item
  
  \item
  
  \item
  
  \item
  
  \item
  
  \item
  
  \item
  
  \item 证明:(\tmtextbf{$\tmmathbf{u}
  \tmmathbf{\cdot}$}$\nabla$)$\tmmathbf{u}$=$\frac{1}{2} \nabla
  (\tmmathbf{u^2}) - \tmmathbf{u \times} (\nabla \times \tmmathbf{u})$
\end{enumeratenumeric}

\section{直角、圆柱、球坐标系下的散度、梯度、旋度}

\subsection{直角坐标系下的散度、梯度、旋度}

\

直角坐标:$(x, y, z)$
\begin{equation}
  \nabla = \tmmathbf{e}_x \tmmathbf{} \frac{\partial}{\partial x} +
  \tmmathbf{e_{}}_y \frac{\partial}{\partial y} + \tmmathbf{e_{}}_z
  \frac{\partial}{\partial z}
\end{equation}
\begin{equation}
  \left(\begin{array}{ccc}
    \frac{\partial \tmmathbf{e_x}}{\partial x} = 0 & \frac{\partial
    \tmmathbf{e_y}}{\partial x} = 0 & \frac{\partial \tmmathbf{e_z}}{\partial
    x} = 0\\
    \frac{\partial \tmmathbf{e_x}}{\partial y} = 0 & \frac{\partial
    \tmmathbf{e_y}}{\partial y} = 0 & \frac{\partial \tmmathbf{e_z}}{\partial
    y} = 0\\
    \frac{\partial \tmmathbf{e_x}}{\partial z} = 0 & \frac{\partial
    \tmmathbf{e_y}}{\partial z} = 0 & \frac{\partial \tmmathbf{e_z}}{\partial
    z} = 0
  \end{array}\right)
\end{equation}
\begin{enumerateroman}
  
  \item 散度:
  \begin{eqnarray}
    \tmop{div} \tmmathbf{f} & = & \nabla \tmmathbf{\cdot} \tmmathbf{f} \\
    & = & \left( \tmmathbf{e}_x \tmmathbf{} \frac{\partial}{\partial x} +
    \tmmathbf{e_{}}_y \frac{\partial}{\partial y} + \tmmathbf{e_{}}_z
    \frac{\partial}{\partial z} \right) \cdot (\tmmathbf{e_{}}_x f_x +
    \tmmathbf{e_{}}_y f_y + \tmmathbf{e_{}}_y f_z) \\
    & = & \frac{\partial f_x}{\partial x} + \frac{\partial f_y}{\partial y} +
    \frac{\partial f_z}{\partial z} 
  \end{eqnarray}
  \item 旋度:
  \begin{eqnarray}
    \tmop{rot} \tmmathbf{f} & = & \nabla \times \tmmathbf{f} \\
    & = & \left( \tmmathbf{e}_x \tmmathbf{} \frac{\partial}{\partial x} +
    \tmmathbf{e_{}}_y \frac{\partial}{\partial y} + \tmmathbf{e_{}}_z
    \frac{\partial}{\partial z} \right) \times (\tmmathbf{e_{}}_x f_x +
    \tmmathbf{e_{}}_y f_y + \tmmathbf{e_{}}_y f_z) \\
    & = & \left|\begin{array}{ccc}
      \tmmathbf{e_x} & \tmmathbf{e_y} & \tmmathbf{e_z}\\
      \frac{\partial}{\partial x} & \frac{\partial}{\partial y} &
      \frac{\partial}{\partial z}\\
      f_x & f_y & f_z
    \end{array}\right| 能写成这样是因为: \left(\begin{array}{ccc}
      \frac{\partial \tmmathbf{e_x}}{\partial x} = 0 & \frac{\partial
      \tmmathbf{e_y}}{\partial x} = 0 & \frac{\partial
      \tmmathbf{e_z}}{\partial x} = 0\\
      \frac{\partial \tmmathbf{e_x}}{\partial y} = 0 & \frac{\partial
      \tmmathbf{e_y}}{\partial y} = 0 & \frac{\partial
      \tmmathbf{e_z}}{\partial y} = 0\\
      \frac{\partial \tmmathbf{e_x}}{\partial z} = 0 & \frac{\partial
      \tmmathbf{e_y}}{\partial z} = 0 & \frac{\partial
      \tmmathbf{e_z}}{\partial z} = 0
    \end{array}\right) 成立 \nonumber\\
    & = & \left( \frac{\partial f_z}{\partial y} - \frac{\partial
    f_y}{\partial z} \right) \tmmathbf{e_x} + \left( \frac{\partial
    f_x}{\partial z} - \frac{\partial f_z}{\partial x} \right) \tmmathbf{e_y}
    + \left( \frac{\partial f_y}{\partial x} - \frac{\partial f_x}{\partial y}
    \right) \tmmathbf{e_z} 
  \end{eqnarray}
  \item 梯度:
  \begin{eqnarray}
    \tmop{grad} \varphi & = & \nabla \varphi \nonumber\\
    & = & \left( \tmmathbf{e_x} \frac{\partial}{\partial x} + \tmmathbf{e_x}
    \frac{\partial}{\partial y} + \tmmathbf{e_z} \frac{\partial}{\partial z}
    \right) \varphi \nonumber\\
    & = & \tmmathbf{e_x} \frac{\partial \varphi}{\partial x} + \tmmathbf{e_y}
    \frac{\partial \varphi}{\partial y} \tmmathbf{+} \tmmathbf{e_z}
    \frac{\partial \varphi}{\partial z} 
  \end{eqnarray}
  \item 拉普拉斯算符:
  \begin{equation}
    \nabla^2 \varphi = \frac{\partial^2 \varphi}{\partial x^2} +
    \frac{\partial^2 \varphi}{\partial y^2} + \frac{\partial^2
    \varphi}{\partial z^2}
  \end{equation}
\end{enumerateroman}

\subsection{柱坐标系下的散度、梯度、旋度}

\

圆柱坐标:($R$,$\phi$,$z$)
\begin{equation}
  \nabla = \tmmathbf{e_R} \frac{\partial}{\partial R} + \tmmathbf{e_{\phi}}
  \frac{1}{R} \frac{\partial}{\partial \phi} + \tmmathbf{e_z}
  \frac{\partial}{\partial z}
\end{equation}


圆柱坐标($R$,$\phi$,$z$)和直角坐标($x, y,
z$)的变换关系为:
\[ \left(\begin{array}{c}
     \tmmathbf{e_R} = \cos \phi \tmmathbf{e_x} + \sin \phi \tmmathbf{e_y}\\
     \tmmathbf{e_{\phi}} = - \sin \phi \tmmathbf{e_x} + \cos \tmmathbf{e_y}\\
     \tmmathbf{e_z} = \tmmathbf{e_z}
   \end{array}\right) \Longrightarrow \left(\begin{array}{c}
     \tmmathbf{e_x} = \cos \phi \tmmathbf{e_R} - \sin \phi
     \tmmathbf{e_{\phi}}\\
     \tmmathbf{e_y} = \sin \phi \tmmathbf{e_R} + \cos \phi
     \tmmathbf{e_{\phi}}\\
     \tmmathbf{e_z} = \tmmathbf{e_z}
   \end{array}\right) \]
\begin{eqnarray*}
  \frac{\partial \tmmathbf{e_{}}_R}{\partial \phi} & = & \frac{\partial (\cos
  \phi \tmmathbf{e_x} + \sin \phi \tmmathbf{e_y})}{\partial \phi}\\
  & = & - \sin \phi \tmmathbf{e_x} + \cos \phi \tmmathbf{e_y}\\
  & = & - \sin \phi (\cos \phi \tmmathbf{e_R} - \sin \phi
  \tmmathbf{e_{\phi}}) + \cos \phi (\sin \phi \tmmathbf{e_R} + \cos \phi
  \tmmathbf{e_{\phi}})\\
  & = & \tmmathbf{e_{\phi}}
\end{eqnarray*}


同理,可得如下矩阵:
\begin{equation}\label{eq:eq46}
  \left|\begin{array}{ccc}
    \frac{\partial \tmmathbf{e_{}}_R}{\partial R} = 0 \quad
    \tmcolor{red}{\checked} & \frac{\partial \tmmathbf{e_{}}_{\phi}}{\partial
    R} = 0 \quad \tmcolor{red}{\checked} & \frac{\partial
    \tmmathbf{e_z}}{\partial R} = 0 \quad \tmcolor{red}{\checked}\\
    \frac{\partial \tmmathbf{e_{}}_R}{\partial \phi} = \tmmathbf{e_{\phi}}
    \quad \tmcolor{red}{\checked} & \frac{\partial
    \tmmathbf{e_{}}_{\phi}}{\partial \phi} = - \tmmathbf{e_R} \quad
    \tmcolor{red}{\checked} & \frac{\partial \tmmathbf{e_z}}{\partial \phi} =
    0 \quad \tmcolor{red}{\checked}\\
    \frac{\partial \tmmathbf{e_{}}_R}{\partial z} = 0 \quad \tmcolor{red}{}
    \tmcolor{red}{\checked} & \frac{\partial \tmmathbf{e_{}}_{\phi}}{\partial
    z} = 0 \quad \tmcolor{red}{\checked} & \frac{\partial
    \tmmathbf{e_z}}{\partial z} = 0 \quad \tmcolor{red}{\checked}
  \end{array}\right|
\end{equation}

\begin{enumerateroman}
  \item 梯度:
  \begin{eqnarray}
    \tmop{grad} \varphi & = & \nabla \varphi \nonumber\\
    & = & \frac{\partial \varphi}{\partial R} \tmmathbf{e_R} + \frac{1}{R}
    \frac{\partial \varphi}{\partial \phi} \tmmathbf{e_{\phi}} +
    \frac{\partial \varphi}{\partial z} \tmmathbf{e_z} 
  \end{eqnarray}
  
  \item 散度:
  \begin{subequations}
  \begin{align} 
    \tmop{div} \tmmathbf{f} & =  \nabla \cdot \tmmathbf{f} \nonumber\\
& =  \left( \tmmathbf{e_R} \frac{\partial}{\partial R} +
\tmmathbf{e_{\phi}} \frac{1}{R} \frac{\partial}{\partial \phi} +
\tmmathbf{e_z} \frac{\partial}{\partial z} \right) \cdot (f_R
\tmmathbf{e_R} + f_{\phi} \tmmathbf{e_{\phi}} + f_z \tmmathbf{e_z})
\nonumber\\
& =  \tmcolor{magenta}{(\tmmathbf{e_R} \cdot \tmmathbf{e_R}) \cdot
	\frac{\partial f_R}{\partial R} + \left( \tmmathbf{e_R} \cdot
	\frac{\partial \tmmathbf{e_R}}{\partial R} \right) \cdot f_R +
	(\tmmathbf{e_R} \cdot \tmmathbf{e_{\phi}}) \frac{\partial
		f_{\phi}}{\partial R} + \left( \tmmathbf{e_R} \cdot \frac{\partial
		\tmmathbf{e_{\phi}}}{\partial R} \right) \cdot f_{\phi} + (\tmmathbf{e_R}
	\cdot \tmmathbf{e_z}) \frac{\partial f_z}{\partial R} + \left(
	\tmmathbf{e_R} \cdot \frac{\partial \tmmathbf{e_z}}{\partial R} \right)
	\cdot f_z}  \\
& +  \tmcolor{blue}{\frac{1}{R} (\tmmathbf{e_{\phi}} \cdot
	\tmmathbf{e_R}) \frac{\partial f_R}{\partial \phi} + \frac{1}{R} \left(
	\tmmathbf{e_{\phi}} \cdot \frac{\partial \tmmathbf{e_R}}{\partial \phi}
	\right) f_R   + \frac{1}{R} (\tmmathbf{e_{\phi}} \cdot \tmmathbf{e_{\phi}})
	\frac{\partial f_{\phi}}{\partial \phi}}  \\
&   \tmcolor{blue}{ + \frac{1}{R} \left(
	\tmmathbf{e_{\phi}} \cdot \frac{\partial \tmmathbf{e_{\phi}}}{\partial
		\phi} \right) f_{\phi} +  \frac{1}{R} (\tmmathbf{e_{\phi}} \cdot
	\tmmathbf{e_z}) \frac{\partial f_z}{\partial \phi} + \frac{1}{R} \left(
	\tmmathbf{e_{\phi}} \cdot \frac{\partial \tmmathbf{e_z}}{\partial \phi}
	\right) f_z}  \\
& +  \tmcolor{red}{(\tmmathbf{e_z} \cdot \tmmathbf{e_R}) \cdot  \label{eq51d}
	\frac{\partial f_R}{\partial z} + \left( \tmmathbf{e_z} \cdot
	\frac{\partial \tmmathbf{e_R}}{\partial z} \right) \cdot f_R +
	(\tmmathbf{e_z} \cdot \tmmathbf{e_{\phi}}) \cdot \frac{\partial
		f_{\phi}}{\partial z} + \left( \tmmathbf{e_z} \cdot \frac{\partial
		\tmmathbf{e_{\phi}}}{\partial z} \right) \cdot f_{\phi} + (\tmmathbf{e_z}
	\cdot \tmmathbf{e_z}) \cdot \frac{\partial f_z}{\partial z} + \left(
	\tmmathbf{e_z} \cdot \frac{\partial \tmmathbf{e_z}}{\partial z} \right)
	\cdot f_z}  \\
& =  \tmcolor{magenta}{\frac{\partial f_R}{\partial R}} +  \label{eq51e}
\tmcolor{blue}{\frac{f_R}{R} + \frac{1}{R} \frac{\partial
		f_{\phi}}{\partial \phi}} + \tmcolor{red}{\frac{\partial f_z}{\partial z}} \\   
& =  \frac{1}{R} \frac{\partial (\tmop{Rf}_R)}{\partial R} + \frac{1}{R}
\frac{\partial f_{\phi}}{\partial \phi} + \frac{\partial f_z}{\partial z} 
  \end{align}
\end{subequations}
  式 (\ref{eq51d})到式 (\ref{eq51e})的化简用到了式 (\ref{eq:eq46})

  \item 旋度:
  \begin{eqnarray}
    \tmop{rot} \tmmathbf{f} & = & \nabla \times \tmmathbf{f} \nonumber\\
    & = & \left( \tmmathbf{e_R} \frac{\partial}{\partial R} +
    \tmmathbf{e_{\phi}} \frac{1}{R} \frac{\partial}{\partial \phi} +
    \tmmathbf{e_z} \frac{\partial}{\partial z} \right) \times (f_R
    \tmmathbf{e_R} + f_{\phi} \tmmathbf{e_{\phi}} + f_z \tmmathbf{e_z})
    \nonumber\\
    & = & \tmcolor{red}{\tmmathbf{e_R} \frac{\partial}{\partial R} \times
    (f_R \tmmathbf{e_R}) + \tmmathbf{e_R} \frac{\partial}{\partial R} \times
    (f_{\phi} \tmmathbf{e_{\phi}}) + \tmmathbf{e_R} \frac{\partial}{\partial
    R} \times (f_z \tmmathbf{e_z})} \nonumber\\
    & + & \tmcolor{brown}{{\color[HTML]{00AA00}\tmmathbf{e_{\phi}}
    \frac{1}{R} \frac{\partial}{\partial \phi} \times (f_R \tmmathbf{e_R}) +
    \tmmathbf{e_{\phi}} \frac{1}{R} \frac{\partial}{\partial \phi} \times
    (f_{\phi} \tmmathbf{e_{\phi}}) + \tmmathbf{e_{\phi}} \frac{1}{R}
    \frac{\partial}{\partial \phi} \times (f_z \tmmathbf{e_z})}} \nonumber\\
    & + & \tmcolor{blue}{\tmmathbf{e_z} \frac{\partial}{\partial z} \times
    (f_R \tmmathbf{e_R}) + \tmmathbf{e_z} \frac{\partial}{\partial z} \times
    (f_{\phi} \tmmathbf{e_{\phi}}) + \tmmathbf{e_z} \frac{\partial}{\partial
    z} \times (f_z \tmmathbf{e_z})} \nonumber\\
    &  &
    注意,这里的叉乘不能和直角坐标系一样写成行列式求解,因为(49)式
    \nonumber\\
    & = & \tmcolor{red}{(\tmmathbf{e_R} \times \tmmathbf{e_R}) \frac{\partial
    f_R}{\partial R} + \left( \tmmathbf{e_R} \times \frac{\partial
    \tmmathbf{e_R}}{\partial R} \right) f_R + (\tmmathbf{e_R} \times
    \tmmathbf{e_{\phi}}) \frac{\partial f_{\phi}}{\partial R}} \nonumber\\ 
    & + &  \tmcolor{red}{\left(
    \tmmathbf{e_R} \times \frac{\partial \tmmathbf{e_{\phi}}}{\partial R}
    \right) f_{\phi} + (\tmmathbf{e_R} \times \tmmathbf{e_z}) \frac{\partial
    f_z}{\partial R} + \left( \tmmathbf{e_R} \times \frac{\partial
    \tmmathbf{e_z}}{\partial R} \right) f_z} \nonumber\\
    & + & {\color[HTML]{00AA00}\frac{1}{R} (\tmmathbf{e_{\phi}} \times
    \tmmathbf{e_R}) \frac{\partial f_R}{\partial \phi} + \frac{1}{R} \left(
    \tmmathbf{e_{\phi}} \times \frac{\partial \tmmathbf{e_R}}{\partial \phi}
    \right) f_R + \frac{1}{R} (\tmmathbf{e_{\phi}} \times \tmmathbf{e_{\phi}})
    \frac{\partial f_{\phi}}{\partial \phi}}  \nonumber\\
    & + & {\color[HTML]{00AA00}\frac{1}{R} \left(
    \tmmathbf{e_{\phi}} \times \frac{\partial \tmmathbf{e_{\phi}}}{\partial
    \phi} \right) f_{\phi} + \frac{1}{R} (\tmmathbf{e_{\phi}} \times
    \tmmathbf{e_z}) \frac{\partial f_z}{\partial \phi} + \frac{1}{R} \left(
    \tmmathbf{e_{\phi}} \times \frac{\partial \tmmathbf{e_z}}{\partial \phi}
    \right) f_z} \nonumber\\
    & + & \tmcolor{blue}{(\tmmathbf{e_z} \times \tmmathbf{e_R})
    \frac{\partial f_R}{\partial z} + \left( \tmmathbf{e_z} \times
    \frac{\partial \tmmathbf{e_R}}{\partial z} \right) f_R + (\tmmathbf{e_z}
    \times \tmmathbf{e_{\phi}}) \frac{\partial f_{\phi}}{\partial z}}  \nonumber\\
	& + & \tmcolor{blue}{\left(
    \tmmathbf{e_z} \times \frac{\partial \tmmathbf{e_{\phi}}}{\partial z}
    \right) f_{\phi} + (\tmmathbf{e_z} \times \tmmathbf{e_z}) \frac{\partial
    f_z}{\partial z} + \left( \tmmathbf{e_z} \times \frac{\partial
    \tmmathbf{e_z}}{\partial z} \right) f_z} \nonumber\\
    &  &  \nonumber\\
    & = & \tmcolor{red}{\frac{\partial f_{\phi}}{\partial R} \tmmathbf{e_z} -
    \frac{\partial f_z}{\partial R} \tmmathbf{e_{\phi}}}
    {\color[HTML]{00AA00}- \frac{1}{R} \frac{\partial f_R}{\partial \phi}
    \tmmathbf{e_z} + \frac{f_{\phi}}{R} \tmmathbf{e_z} + \frac{1}{R}
    \frac{\partial f_z}{\partial \phi} \tmmathbf{e_R}} +
    \tmcolor{blue}{\frac{\partial f_R}{\partial z} \tmmathbf{e_{\phi}} -
    \frac{\partial f_{\phi}}{\partial z} \tmmathbf{e_R}} \nonumber\\
    & = & \left( \frac{1}{R} \frac{\partial f_z}{\partial \phi} -
    \frac{\partial f_{\phi}}{\partial z} \right) \tmmathbf{e_R} + \left(
    \frac{\partial f_R}{\partial z} - \frac{\partial f_z}{\partial R} \right)
    \tmmathbf{e_{\phi}} + \left( \frac{1}{R} \frac{\partial
    (\tmop{Rf}_{\phi})}{\partial R} - \frac{1}{R} \frac{\partial f_R}{\partial
    \phi} \right) \tmmathbf{e_z} 
  \end{eqnarray}
  \item 拉普拉斯算符:
  \begin{eqnarray}
    \Delta \varphi & = & \nabla^2 \varphi \nonumber\\
    & = & \nabla \cdot (\nabla \varphi) \nonumber\\
    & = & \left( \tmmathbf{e_R} \frac{\partial}{\partial R} +
    \tmmathbf{e_{\phi}} \frac{1}{R} \frac{\partial}{\partial \phi} +
    \tmmathbf{e_z} \frac{\partial}{\partial z} \right) \cdot \left(
    \frac{\partial \varphi}{\partial R} \tmmathbf{e_R} + \frac{1}{R}
    \frac{\partial \varphi}{\partial \phi} \tmmathbf{e_{\phi}} +
    \frac{\partial \varphi}{\partial z} \tmmathbf{e_z} \right) \nonumber\\
    & = & \tmcolor{red}{\tmmathbf{e_R} \frac{\partial}{\partial R} \cdot
    \left( \frac{\partial \varphi}{\partial R} \tmmathbf{e_R} \right) +
    \tmmathbf{e_R} \frac{\partial}{\partial R} \cdot \left( \frac{1}{R}
    \frac{\partial \varphi}{\partial \phi} \tmmathbf{e_{\phi}} \right) +
    \tmmathbf{e_R} \frac{\partial}{\partial R} \cdot \left( \frac{\partial
    \varphi}{\partial z} \tmmathbf{e_z} \right)} \nonumber\\
    & + & \tmcolor{blue}{\tmmathbf{e_{\phi}} \frac{1}{R}
    \frac{\partial}{\partial \phi} \cdot \left( \frac{\partial
    \varphi}{\partial R} \tmmathbf{e_R} \right) + \tmmathbf{e_{\phi}}
    \frac{1}{R} \frac{\partial}{\partial \phi} \cdot \left( \frac{1}{R}
    \frac{\partial \varphi}{\partial \phi} \tmmathbf{e_{\phi}} \right) +
    \tmmathbf{e_{\phi}} \frac{1}{R} \frac{\partial}{\partial \phi} \cdot
    \left( \frac{\partial \varphi}{\partial z} \tmmathbf{e_z} \right)}
    \nonumber\\
    & + & \tmmathbf{e_z} \frac{\partial}{\partial z} \cdot \left(
    \frac{\partial \varphi}{\partial R} \tmmathbf{e_R} \right) +
    \tmmathbf{e_z} \frac{\partial}{\partial z} \cdot \left( \frac{1}{R}
    \frac{\partial \varphi}{\partial \phi} \tmmathbf{e_{\phi}} \right) +
    \tmmathbf{e_z} \frac{\partial}{\partial z} \cdot \left( \frac{\partial
    \varphi}{\partial z} \tmmathbf{e_z} \right) \nonumber\\
    &  &  \nonumber\\
    & = & \tmcolor{red}{(\tmmathbf{e_R} \cdot \tmmathbf{e_R})
    \frac{\partial^2 \varphi}{\partial R^2} + \left( \tmmathbf{e_R} \cdot
    \frac{\partial e_R}{\partial R} \right) \frac{\partial \varphi}{\partial
    R} + (\tmmathbf{e_R} \cdot \tmmathbf{e_{\phi}}) \frac{1}{R}
    \frac{\partial^2 \varphi}{\partial R \partial \phi} + \left(
    \tmmathbf{e_R} \cdot \frac{\partial \tmmathbf{e_{\varphi}}}{\partial R}
    \right) \frac{1}{R} \frac{\partial \varphi}{\partial \phi}}  \nonumber\\
	& + &   \tmcolor{red}{(\tmmathbf{e_R} \cdot \tmmathbf{e_{\phi}}) \frac{\partial
    \varphi}{\partial \phi} \frac{\partial}{\partial R} \left( \frac{1}{R}
    \right) + (\tmmathbf{e_R} \cdot \tmmathbf{e_z}) \frac{\partial^2
    \varphi}{\partial R \partial z} + \left( \tmmathbf{e_R} \cdot
    \frac{\partial \tmmathbf{e_z}}{\partial R} \right) \frac{\partial
    \varphi}{\partial z}} \nonumber\\
    & + & \tmcolor{blue}{(\tmmathbf{e_{\phi}} \cdot \tmmathbf{e_R})
    \frac{1}{R} \frac{\partial^2 \varphi}{\partial \phi \partial R} + \left(
    \tmmathbf{e_{\phi}} \cdot \frac{\partial \tmmathbf{e_R}}{\partial \phi}
    \right) \frac{1}{R} \frac{\partial \varphi}{\partial R} +
    (\tmmathbf{e_{\phi}} \cdot \tmmathbf{e_{\phi}}) \frac{1}{R^2}
    \frac{\partial^2 \varphi}{\partial \phi^2} + \left( \tmmathbf{e_{\phi}}
    \cdot \frac{\partial \tmmathbf{e_{\phi}}}{\partial \phi} \right)
    \frac{1}{R^2} \frac{\partial \varphi}{\partial \phi}} \nonumber\\
  	& + & \tmcolor{blue}{(\tmmathbf{e_{\phi}} \cdot \tmmathbf{e_{\phi}}) \frac{1}{R} \frac{\partial
    \varphi}{\partial \phi} \frac{\partial}{\partial \phi} \left( \frac{1}{R}
    \right) + (\tmmathbf{e_{\phi}} \cdot \tmmathbf{e_z}) \frac{1}{R}
    \frac{\partial^2 \varphi}{\partial \phi \partial z} + \left(
    \tmmathbf{e_{\phi}} \cdot \frac{\partial \tmmathbf{e_z}}{\partial \phi}
    \right) \frac{1}{R} \frac{\partial \varphi}{\partial z}} \nonumber\\
    & + & (\tmmathbf{e_z} \cdot \tmmathbf{e_R}) \frac{\partial^2
    \varphi}{\partial z \partial R} + \left( \tmmathbf{e_z} \cdot
    \frac{\partial \tmmathbf{e_R}}{\partial z} \right) \frac{\partial
    \varphi}{\partial R} + (\tmmathbf{e_z} \cdot \tmmathbf{e_{\phi}})
    \frac{1}{R} \frac{\partial^2 \varphi}{\partial z \partial \phi} + \left(
    \tmmathbf{e_z} \cdot \frac{\partial \tmmathbf{e_{\phi}}}{\partial z}
    \right) \frac{1}{R} \frac{\partial \varphi}{\partial \phi} \nonumber\\
    & + & (\tmmathbf{e_z} \cdot \tmmathbf{e_{\phi}}) \frac{\partial
    \varphi}{\partial \phi} \frac{\partial}{\partial z} \left( \frac{1}{R}
    \right) + (\tmmathbf{e_z} \cdot \tmmathbf{e_z}) \frac{\partial^2
    \varphi}{\partial z^2} + \left( \tmmathbf{e_z} \cdot \frac{\partial
    \tmmathbf{e_z}}{\partial z} \right) \frac{\partial^{} \varphi}{\partial z}
    \nonumber\\
    &  &  \nonumber\\
    & = & \frac{\partial^2 \varphi}{\partial R^2} + \frac{1}{R}
    \frac{\partial \varphi}{\partial R} + \frac{1}{R^2} \frac{\partial^2
    \varphi}{\partial \phi^2} + \frac{\partial^2 \varphi}{\partial z^2}
    \nonumber\\
    & = & \frac{1}{R} \frac{\partial}{\partial R} \left( R \frac{\partial
    \varphi}{\partial R} \right) + \frac{1}{R^2} \frac{\partial^2
    \varphi}{\partial \phi^2} + \frac{\partial^2 \varphi}{\partial z^2} 
  \end{eqnarray}
\end{enumerateroman}

\subsection{球坐标系下的散度、梯度、旋度}

\

球坐标:($r, \theta, \phi$)
\begin{equation}
  \nabla = \tmmathbf{e_r} \frac{\partial}{\partial r} + \tmmathbf{e_{\theta}}
  \frac{1}{r} \frac{\partial}{\partial \theta} + \tmmathbf{e_{\phi}}
  \frac{1}{\tmop{rsin} \theta} \frac{\partial}{\partial \phi}
\end{equation}


球坐标($r, \theta, \phi$)和直角坐标($x, y, z$)的变换关系为:
\[ \begin{array}{c}
     \tmmathbf{e_r} = \sin \theta \cos \phi \tmmathbf{e_x} + \sin \theta \sin
     \phi \tmmathbf{e_y} + \cos \theta \tmmathbf{e_z}\\
     \tmmathbf{e_{\theta}} = \cos \theta \cos \phi \tmmathbf{e_x} + \cos
     \theta \sin \phi \tmmathbf{e_y} - \sin \theta \tmmathbf{e_z}\\
     \tmmathbf{e_{\phi}} = - \sin \phi \tmmathbf{e_x} + \cos \phi
     \tmmathbf{e_y}
   \end{array} \]
\begin{eqnarray*}
  \left(\begin{array}{c}
    \tmmathbf{e_x} = \sin \theta \cos \phi \tmmathbf{e_r} + \cos \theta \cos
    \phi \tmmathbf{e_{\theta}} - \sin \phi \tmmathbf{e_{\phi}}\\
    \tmmathbf{e_y} = \sin \theta \sin \phi \tmmathbf{e_r} + \cos \theta \sin
    \phi \tmmathbf{e_{\theta}} + \cos \phi \tmmathbf{e_{\phi}}\\
    \tmmathbf{e_z} = \cos \theta \tmmathbf{e_r} - \sin \theta
    \tmmathbf{e_{\theta}}
  \end{array}\right) &  & 
\end{eqnarray*}


由上述两组变换关系可得:
\begin{equation} \label{eq55}
  \begin{array}{ccc}
    \frac{\partial \tmmathbf{e_r}}{\partial r} = 0 & \frac{\partial
    \tmmathbf{e_r}}{\partial \theta} = \tmmathbf{e_{\theta}} & \frac{\partial
    \tmmathbf{e_r}}{\partial \phi} = \sin \theta \tmmathbf{e_{\phi}}\\
    \frac{\partial \tmmathbf{e_{\theta}}}{\partial r} = 0 & \frac{\partial
    \tmmathbf{e_{\theta}}}{\partial \theta} = - \tmmathbf{e_r} &
    \frac{\partial \tmmathbf{e_{\theta}}}{\partial \phi} = \cos \theta
    \tmmathbf{e_{\phi}}\\
    \frac{\partial \tmmathbf{e_{\phi}}}{\partial r} = 0 & \frac{\partial
    \tmmathbf{e_{\phi}}}{\partial \theta} = 0 & \frac{\partial
    \tmmathbf{e_{\phi}}}{\partial \phi} = - \sin \theta \tmmathbf{e_r} - \cos
    \theta \tmmathbf{e_{\theta}}
  \end{array}
\end{equation}
\begin{enumerateroman}
  \item 梯度:
  \begin{eqnarray}
    \tmop{grad} \varphi & = & \nabla \varphi \nonumber\\
    & = & \tmmathbf{e_r} \frac{\partial \varphi}{\partial r} +
    \tmmathbf{e_{\theta}} \frac{1}{r} \frac{\partial \varphi}{\partial \theta}
    + \tmmathbf{e_{\phi}} \frac{1}{\tmop{rsin} \theta} \frac{\partial
    \varphi}{\partial \phi} 
  \end{eqnarray}
  \item 散度:
  \begin{subequations}
  \begin{eqnarray}
    \tmop{div} \tmmathbf{f} & = & \nabla \cdot \tmmathbf{f} \nonumber\\
    & = & \left( \tmmathbf{e_r} \frac{\partial}{\partial r} +
    \tmmathbf{e_{\theta}} \frac{1}{r} \frac{\partial}{\partial \theta} +
    \tmmathbf{e_{\phi}} \frac{1}{\tmop{rsin} \theta} \frac{\partial}{\partial
    \phi} \right) \cdot (\tmmathbf{e_r} f_r + \tmmathbf{e_{\theta}} f_{\theta}
    + \tmmathbf{e_{\phi}} f_{\phi})  \\
    & = & \tmmathbf{e_r} \frac{\partial}{\partial r} \cdot (\tmmathbf{e_r}  \label{eq57a}
    f_r) + \tmmathbf{e_r} \frac{\partial}{\partial r} \cdot
    (\tmmathbf{e_{\theta}} f_{\theta}) + \tmmathbf{e_r}
    \frac{\partial}{\partial r} \cdot (\tmmathbf{e_{\phi}} f_{\phi})
     \nonumber \\
    & + & \tmmathbf{e_{\theta}} \frac{1}{r} \frac{\partial}{\partial \theta}
    \cdot (\tmmathbf{e_r} f_r) + \tmmathbf{e_{\theta}} \frac{1}{r}
    \frac{\partial}{\partial \theta} \cdot (\tmmathbf{e_{\theta}} f_{\theta})
    + \tmmathbf{e_{\theta}} \frac{1}{r} \frac{\partial}{\partial \theta} \cdot
    (\tmmathbf{e_{\phi}} f_{\phi}) \nonumber \\
    & + & \tmmathbf{e_{\phi}} \frac{1}{\tmop{rsin} \theta}
    \frac{\partial}{\partial \phi} \cdot (\tmmathbf{e_r} f_r) +
    \tmmathbf{e_{\phi}} \frac{1}{\tmop{rsin} \theta} \frac{\partial}{\partial
    \phi} \cdot (\tmmathbf{e_{\theta}} f_{\theta}) + \tmmathbf{e_{\phi}}
    \frac{1}{\tmop{rsin} \theta} \frac{\partial}{\partial \phi} \cdot  
    (\tmmathbf{e_{\phi}} f_{\phi}) \nonumber \\
    & = & \frac{1}{r^2} \frac{\partial (r^2 f_r)}{\partial r} +
    \frac{1}{\tmop{rsin} \theta} \frac{\partial (f_{\theta} \cdot \sin
    \theta)}{\partial \theta} + \frac{1}{\tmop{rsin} \theta} \frac{\partial
    f_{\phi}}{\partial \phi}   \label{eq57b}
  \end{eqnarray}
\end{subequations}
式(\ref{eq57a})到式(\ref{eq57b})的化简利用了式(\ref{eq55})

  \item 旋度:
  \begin{eqnarray}
    \tmop{rot} \tmmathbf{f} & = & \nabla \times \tmmathbf{f} \nonumber\\
    \nonumber\\
    & = & \frac{1}{\tmop{rsin} \theta} \left( \frac{\partial (f_{\phi} \sin
    \theta)}{\partial \theta} - \frac{\partial f_{\theta}}{\partial \phi}
    \right) \tmmathbf{e_r} + \frac{1}{r} \left( \frac{1}{\sin \theta}
    \frac{\partial f_r}{\partial \phi} - \frac{\partial
    (\tmop{rf}_{\phi})}{\partial r} \right) \tmmathbf{e_{\theta}} +
    \frac{1}{r} \left( \frac{\partial (\tmop{rf}_{\theta})}{\partial r} -
    \frac{\partial f_r}{\partial \theta} \right) \tmmathbf{e_{\phi}} 
  \end{eqnarray}
  化简利用了式(\ref{eq55})
  
  \item 拉普拉斯算符:
  \begin{eqnarray}
    \Delta \varphi & = & \nabla^2 \varphi \nonumber\\
    & = & \nabla \cdot (\nabla \varphi) \nonumber\\
    & = & \frac{1}{r^2} \frac{\partial}{\partial r} \left( r^2 \frac{\partial
    \varphi}{\partial r} \right) + \frac{1}{r^2 \sin \theta}
    \frac{\partial}{\partial \theta} \left( \sin \theta \frac{\partial
    \varphi}{\partial \theta} \right) + \frac{1}{r^2 \sin^2 \theta}
    \frac{\partial^2 \varphi}{\partial \phi^2} 
  \end{eqnarray}
\end{enumerateroman}

\subsection{环坐标系下的散度、梯度、旋度}

\

环坐标系:$(r, \theta, \varphi) ,单位基失 (\tmmathbf{e_r},
\tmmathbf{e_{\theta}}, \tmmathbf{e_{\varphi}})$,$\tmmathbf{e_{\varphi}} =
\tmmathbf{e_r } \times \tmmathbf{e_{\theta}}$
\begin{equation}
  \nabla = \tmmathbf{e_r} \frac{\partial}{\partial r} + \tmmathbf{e_{\theta}}
  \frac{1}{r} \frac{\partial}{\partial \theta} + \tmmathbf{e_{\varphi}}
  \frac{1}{R_0} \frac{\partial}{\partial \varphi} ,其中R_0 为
  \tmop{Tokamak} 大半径
\end{equation}


环坐标系$(r, \theta,
\varphi)$和柱坐标系($R$,$\phi$,$z$)的坐标基失变换关系为:
\begin{equation}
  \left|\begin{array}{c}
    \tmmathbf{e_R} = \cos \theta \tmmathbf{e_r} - \sin \theta \tmmathbf{}
    \tmmathbf{e_{\theta}}\\
    \tmmathbf{e_z} = \sin \theta \tmmathbf{e_r} + \cos \theta
    \tmmathbf{e_{\theta}}\\
    \tmmathbf{e_{\phi}} = - \tmmathbf{e_{\varphi}}
  \end{array}\right| \Longrightarrow \left(\begin{array}{c}
    \tmmathbf{e_r} = \cos \theta \tmmathbf{e_R} + \sin \theta \tmmathbf{e_z}\\
    \tmmathbf{e_{\theta}} = - \sin \theta \tmmathbf{e_R} + \cos \theta
    \tmmathbf{e_z}\\
    \tmmathbf{e_{\varphi}} = - \tmmathbf{e_{\phi}}
  \end{array}\right)
\end{equation}


由(23)可以得:
\begin{eqnarray*}
  \frac{\partial \tmmathbf{e_{}}_r}{\partial \theta} & = & \frac{\partial
  (\cos \theta \tmmathbf{e_R} + \sin \theta \tmmathbf{e_z})}{\partial
  \theta}\\
  & = & - \sin \theta \tmmathbf{e_R} + \cos \theta \tmmathbf{e_z}\\
  & = & - \sin \theta (\cos \theta \tmmathbf{e_r} - \sin \theta \tmmathbf{}
  \tmmathbf{e_{\theta}}) + \cos \theta (\sin \theta \tmmathbf{e_r} + \cos
  \theta \tmmathbf{e_{\theta}})\\
  & = & \tmmathbf{e_{\theta}}
\end{eqnarray*}


同理可得如下矩阵:
\begin{equation}\label{eq62}
  \left(\begin{array}{ccc}
    \frac{\partial \tmmathbf{e} \tmmathbf{_{}}_r}{\partial r} = 0 \quad
    \tmcolor{red}{\checked} & \frac{\partial
    \tmmathbf{e_{}}_{\theta}}{\partial r} = 0 \quad \tmcolor{red}{\checked} &
    \frac{\partial \tmmathbf{e_{\varphi}} \tmmathbf{_{}}}{\partial r} = 0
    \quad \tmcolor{red}{\checked}\\
    \frac{\partial \tmmathbf{e_{}}_r}{\partial \theta} = \tmmathbf{e_{\theta}}
    \quad \tmcolor{red}{\checked} & \frac{\partial
    \tmmathbf{e_{}}_{\theta}}{\partial \theta} = - \tmmathbf{e_r} \quad
    \tmcolor{red}{\checked} & \frac{\partial \tmmathbf{e_{\varphi}}
    \tmmathbf{_{}}}{\partial \theta} = 0 \quad \tmcolor{red}{\checked}\\
    \frac{\partial \tmmathbf{e_{}}_r}{\partial \varphi} = \cos \theta \cdot
    \tmmathbf{e_{\varphi}} \quad \tmcolor{red}{\checked} & \frac{\partial
    \tmmathbf{e_{}}_{\theta}}{\partial \varphi} = - \sin \theta \cdot
    \tmmathbf{e_{\varphi}} \quad \tmcolor{red}{\checked} & \frac{\partial
    \tmmathbf{e_{\varphi}} \tmmathbf{_{}}}{\partial \varphi} = \sin \theta
    \cdot \tmmathbf{e_{\theta}} - \cos \theta \cdot \tmmathbf{e_r} \quad
    \tmcolor{red}{\checked}
  \end{array}\right)
\end{equation}

\begin{enumerateroman}
  \item 梯度:
  \begin{eqnarray}
    \tmop{grad} \psi & = & \nabla \psi \nonumber\\
    & = & \left( \tmmathbf{e_r} \frac{\partial}{\partial r} +
    \tmmathbf{e_{\theta}} \frac{1}{r} \frac{\partial}{\partial \theta} +
    \tmmathbf{e_{\varphi}} \frac{1}{R_0} \frac{\partial}{\partial \varphi}
    \right) \psi \nonumber\\
    & = & \tmmathbf{e_r} \frac{\partial \psi}{\partial r} +
    \tmmathbf{e_{\theta}} \frac{1}{r} \frac{\partial \psi}{\partial \theta} +
    \tmmathbf{e_{\varphi}} \frac{1}{R_0} \frac{\partial \psi}{\partial
    \varphi} 
  \end{eqnarray}
  \item 散度:
  \begin{subequations}
  \begin{eqnarray}
    \tmop{div} \tmmathbf{f} & = & \nabla \cdot \tmmathbf{f} \nonumber\\
    & = & \left( \tmmathbf{e_r} \frac{\partial}{\partial r} +
    \tmmathbf{e_{\theta}} \frac{1}{r} \frac{\partial}{\partial \theta} +
    \tmmathbf{e_{\varphi}} \frac{1}{R_0} \frac{\partial}{\partial \varphi}
    \right) \cdot (f_r \tmmathbf{e_r} + f_{\theta} \tmmathbf{e_{\theta}} +
    f_{\varphi} \tmmathbf{e_{\varphi}}) \nonumber\\
    & = & \tmcolor{red}{\tmmathbf{e_r} \frac{\partial}{\partial r} \cdot (f_r
    \tmmathbf{e_r}) + \tmmathbf{e_r} \frac{\partial}{\partial r} \cdot
    (f_{\theta} \tmmathbf{e_{\theta}}) + \tmmathbf{e_r}
    \frac{\partial}{\partial r} \cdot (f_{\varphi} \tmmathbf{e_{\varphi}})}
    \nonumber\\
    & + & \tmcolor{blue}{\tmmathbf{e_{\theta}} \frac{1}{r}
    \frac{\partial}{\partial \theta} \cdot (f_r \tmmathbf{e_r}) +
    \tmmathbf{e_{\theta}} \frac{1}{r} \frac{\partial}{\partial \theta} \cdot
    (f_{\theta} \tmmathbf{e_{\theta}}) + \tmmathbf{e_{\theta}} \frac{1}{r}
    \frac{\partial}{\partial \theta} \cdot (f_{\varphi}
    \tmmathbf{e_{\varphi}})} \nonumber\\
    & + & {\color[HTML]{00AA00}\tmmathbf{e_{\varphi}} \frac{1}{R_0}
    \frac{\partial}{\partial \varphi} \cdot (f_r \tmmathbf{e_r}) +
    \tmmathbf{e_{\varphi}} \frac{1}{R_0} \frac{\partial}{\partial \varphi}
    \cdot (f_{\theta} \tmmathbf{e_{\theta}}) + \tmmathbf{e_{\varphi}}
    \frac{1}{R_0} \frac{\partial}{\partial \varphi} \cdot (f_{\varphi}
    \tmmathbf{e_{\varphi}})} \nonumber\\
    &  &  \nonumber\\
    & = & \tmcolor{red}{(\tmmathbf{e_r} \cdot \tmmathbf{e_r}) \frac{\partial  \label{eq64a}
    f_r}{\partial r} + \left( \tmmathbf{e_r} \cdot \frac{\partial
    \tmmathbf{e_r}}{\partial r} \right) f_r + (\tmmathbf{e_r} \cdot
    \tmmathbf{e_{\theta}}) \frac{\partial f_{\theta}}{\partial r} + \left(
    \tmmathbf{e_r} \cdot \frac{\partial \tmmathbf{e_{\theta}}}{\partial r}
    \right) f_{\theta} + (\tmmathbf{e_r} \cdot \tmmathbf{e_{\varphi}})
    \frac{\partial f_{\varphi}}{\partial r} + \left( \tmmathbf{e_r} \cdot
    \frac{\partial \tmmathbf{e_{\varphi}}}{\partial r} \right) f_{\varphi}}
     \\
    & + & \tmcolor{blue}{(\tmmathbf{e_{\theta}} \cdot \tmmathbf{e_r})
    \frac{1}{r} \frac{\partial f_r}{\partial \theta} + \left(
    \tmmathbf{e_{\theta}} \cdot \frac{\partial \tmmathbf{e_r}}{\partial
    \theta} \right) \frac{1}{r} f_r + (\tmmathbf{e_{\theta}} \cdot
    \tmmathbf{e_{\theta}}) \frac{1}{r} \frac{\partial f_{\theta}}{\partial
    \theta} + \left( \tmmathbf{e_{\theta}} \cdot \frac{\partial
    \tmmathbf{e_{\theta}}}{\partial \theta} \right) \frac{1}{r} f_{\theta} +
    (\tmmathbf{e_{\theta}} \cdot \tmmathbf{e_{\varphi}}) \frac{1}{r}
    \frac{\partial f_{\varphi}}{\partial \theta} + \left(
    \tmmathbf{e_{\theta}} \cdot \frac{\partial
    \tmmathbf{e_{\varphi}}}{\partial \theta} \right) \frac{1}{r} f_{\varphi}}
    \nonumber\\
    & + & {\color[HTML]{00AA00}(\tmmathbf{e_{\varphi}} \cdot \tmmathbf{e_r})
    \frac{1}{R_0} \frac{\partial f_r}{\partial \varphi} + \left(
    \tmmathbf{e_{\varphi}} \cdot \frac{\partial \tmmathbf{e_r}}{\partial
    \varphi} \right) \frac{1}{R_0} f_r + (\tmmathbf{e_{\varphi}} \cdot
    \tmmathbf{e_{\theta}}) \frac{1}{R_0} \frac{\partial f_{\theta}}{\partial
    \varphi}} \nonumber\\
 	& + & {\color[HTML]{00AA00}\left( \tmmathbf{e_{\varphi}} \cdot \frac{\partial
    \tmmathbf{e_{\theta}}}{\partial \varphi} \right) \frac{1}{R_0} f_{\theta}
    + (\tmmathbf{e_{\varphi}} \cdot \tmmathbf{e_{\varphi}}) \frac{1}{R_0}
    \frac{\partial f_{\varphi}}{\partial \varphi} + \left(
    \tmmathbf{e_{\varphi}} \cdot \frac{\partial
    \tmmathbf{e_{\varphi}}}{\partial \varphi} \right) \frac{1}{R_0}
    f_{\varphi}} \nonumber\\
    & = & \tmcolor{red}{\frac{\partial f_r}{\partial r}} +
    \tmcolor{blue}{\frac{f_r}{r} + \frac{1}{r} \frac{\partial
    f_{\theta}}{\partial \theta}} + {\color[HTML]{00AA00}\frac{1}{R_0} f_r
    \cos \theta - \frac{1}{R_0} f_{\theta} \sin \theta + \frac{1}{R_0}
    \frac{\partial f_{\varphi}}{\partial \varphi}} \\  \label{eq64b}
    & = & \frac{1}{r} \frac{\partial (\tmop{rf}_r)}{\partial r} + \frac{1}{r}
    \frac{\partial f_{\theta}}{\partial \theta} + \frac{1}{R_0} \frac{\partial
    f_{\varphi}}{\partial \varphi} + \frac{f_r}{R_0} \cos \theta -
    \frac{f_{\theta}}{R_0} \sin \theta 
  \end{eqnarray}
\end{subequations}
式(\ref{eq64a})到式(\ref{eq64b})的化简利用了式(\ref{eq62})

  \item 旋度:
  \begin{subequations}
  \begin{eqnarray}
    \tmop{rot} \tmmathbf{f} & = & \nabla \times \tmmathbf{f} \nonumber\\
    & = & \left( \tmmathbf{e_r} \frac{\partial}{\partial r} +
    \tmmathbf{e_{\theta}} \frac{1}{r} \frac{\partial}{\partial \theta} +
    \tmmathbf{e_{\varphi}} \frac{1}{R_0} \frac{\partial}{\partial \varphi}
    \right) \times (f_r \tmmathbf{e_r} + f_{\theta} \tmmathbf{e_{\theta}} +
    f_{\varphi} \tmmathbf{e_{\varphi}}) \nonumber\\
    & = & \tmcolor{red}{\tmmathbf{e_r} \frac{\partial}{\partial r} \times
    (f_r \tmmathbf{e_r}) + \tmmathbf{e_r} \frac{\partial}{\partial r} \times
    (f_{\theta} \tmmathbf{e_{\theta}}) + \tmmathbf{e_r}
    \frac{\partial}{\partial r} \times (f_{\varphi} \tmmathbf{e_{\varphi}})}
    \nonumber\\
    & + & \tmcolor{blue}{\tmmathbf{e_{\theta}} \frac{1}{r}
    \frac{\partial}{\partial \theta} \times (f_r \tmmathbf{e_r}) +
    \tmmathbf{e_{\theta}} \frac{1}{r} \frac{\partial}{\partial \theta} \times
    (f_{\theta} \tmmathbf{e_{\theta}}) + \tmmathbf{e_{\theta}} \frac{1}{r}
    \frac{\partial}{\partial \theta} \times (f_{\varphi}
    \tmmathbf{e_{\varphi}})} \nonumber\\
    & + & \tmmathbf{e_{\varphi}} \frac{1}{R_0} \frac{\partial}{\partial
    \varphi} \times (f_r \tmmathbf{e_r}) + \tmmathbf{e_{\varphi}}
    \frac{1}{R_0} \frac{\partial}{\partial \varphi} \times (f_{\theta}
    \tmmathbf{e_{\theta}}) + \tmmathbf{e_{\varphi}} \frac{1}{R_0}
    \frac{\partial}{\partial \varphi} \times (f_{\varphi}
    \tmmathbf{e_{\varphi}}) \nonumber\\
    &  &  \nonumber\\
    & = & \tmcolor{red}{(\tmmathbf{e_r \times} \tmmathbf{e_r}) \frac{\partial  
    f_r}{\partial r} + \left( \tmmathbf{e_r \times} \frac{\partial
    \tmmathbf{e_r}}{\partial r} \right) f_r + (\tmmathbf{e_r \times}
    \tmmathbf{e_{\theta}}) \frac{\partial f_{\theta}}{\partial r} + \left(
    \tmmathbf{e_r \times} \frac{\partial \tmmathbf{e_{\theta}}}{\partial r}
    \right) f_{\theta}}  \\ \label{eq65a}
 	&+& \tmcolor{red}{(\tmmathbf{e_r \times} \tmmathbf{e_{\varphi}})
    \frac{\partial f_{\varphi}}{\partial r} + \left( \tmmathbf{e_r \times}
    \frac{\partial \tmmathbf{e_{\varphi}}}{\partial r} \right) f_{\varphi}}
     \nonumber \\
    & + & \tmcolor{blue}{(\tmmathbf{e_{\theta} \times} \tmmathbf{e_r})
    \frac{1}{r} \frac{\partial f_r}{\partial \theta} + \left(
    \tmmathbf{e_{\theta} \times} \frac{\partial \tmmathbf{e_r}}{\partial
    \theta} \right) \frac{1}{r} f_r + (\tmmathbf{e_{\theta} \times}
    \tmmathbf{e_{\theta}}) \frac{1}{r} \frac{\partial f_{\theta}}{\partial
    \theta}}  \nonumber\\
 	& + & \tmcolor{blue}{\left( \tmmathbf{e_{\theta} \times} \frac{\partial
    \tmmathbf{e_{\theta}}}{\partial \theta} \right) \frac{1}{r} f_{\theta} +
    (\tmmathbf{e_{\theta} \times} \tmmathbf{e_{\varphi}}) \frac{1}{r}
    \frac{\partial f_{\varphi}}{\partial \theta} + \left( \tmmathbf{e_{\theta}
    \times} \frac{\partial \tmmathbf{e_{\varphi}}}{\partial \theta} \right)
    \frac{1}{r} f_{\varphi}} \nonumber\\
    & + & (\tmmathbf{e_{\varphi} \times} \tmmathbf{e_r}) \frac{1}{R_0}
    \frac{\partial f_r}{\partial \varphi} + \left( \tmmathbf{e_{\varphi}
    \times} \frac{\partial \tmmathbf{e_r}}{\partial \varphi} \right)
    \frac{f_r}{R_0} + (\tmmathbf{e_{\varphi} \times} \tmmathbf{e_{\theta}})
    \frac{1}{R_0} \frac{\partial f_{\theta}}{\partial \varphi} \nonumber\\
     & + & \left(
    \tmmathbf{e_{\varphi} \times} \frac{\partial
    \tmmathbf{e_{\theta}}}{\partial \varphi} \right) \frac{f_{\theta}}{R_0} +
    (\tmmathbf{e_{\varphi} \times} \tmmathbf{e_{\varphi}}) \frac{1}{R_0}
    \frac{\partial f_{\varphi}}{\partial \varphi} + \left(
    \tmmathbf{e_{\varphi} \times} \frac{\partial
    \tmmathbf{e_{\varphi}}}{\partial \varphi} \right) \frac{f_{\varphi}}{R_0}
    \nonumber\\
    & = & \tmcolor{red}{\frac{\partial f_{\theta}}{\partial r}  
    \tmmathbf{e_{\varphi}} - \frac{\partial f_{\varphi}}{\partial r}
    \tmmathbf{e_{\theta}}} \tmcolor{blue}{- \frac{\partial f_r}{\partial
    \theta} \frac{e_{\varphi}}{r} + \frac{f_{\theta}}{r}
    \tmmathbf{e_{\varphi}} + \frac{\partial f_{\varphi}}{\partial \theta}
    \frac{\tmmathbf{e_r}}{r}} \nonumber\\
	&+& \frac{\partial f_r}{\partial \varphi}
    \frac{\tmmathbf{e_{\theta}}}{R_0} - \frac{\partial f_{\theta}}{\partial
    \varphi} \frac{e_r}{R_0} + \frac{f_{\varphi}}{R_0} (- \sin \theta
    \tmmathbf{e_r} - \cos \theta \tmmathbf{e_{\theta}})  \\ \label{eq65b}
    & = & \left( \frac{1}{r} \frac{\partial f_{\varphi}}{\partial \theta} -
    \frac{1}{R_0} \frac{\partial f_{\theta}}{\partial \varphi} \right)
    \tmmathbf{e_r} + \left( \frac{1}{R_0} \frac{\partial f_r}{\partial
    \varphi} - \frac{\partial f_{\varphi}}{\partial r} \right)
    \tmmathbf{e_{\theta}} +\left( \frac{1}{r} \frac{\partial
    (\tmop{rf}_{\theta})}{\partial r}  - \frac{1}{r} \frac{\partial
    f_r}{\partial \theta} \right) \tmmathbf{e_z} - \frac{B_{\varphi}}{R_0}
    \sin \theta \tmmathbf{e_r} \nonumber\\ 
    &-& \cos \theta \frac{B_{\varphi}}{R_0}
    \tmmathbf{e_{\theta}} 
  \end{eqnarray}
\end{subequations}
式(\ref{eq65a})到式(\ref{eq65b})的化简利用了式(\ref{eq62})

  \item 拉普拉斯算符:
  \begin{eqnarray}
    \Delta \varphi & = & \nabla^2 \varphi \nonumber\\
    & = & \nabla \cdot (\nabla \varphi) \nonumber\\
    & = & \left( \tmmathbf{e_r} \frac{\partial}{\partial r} +
    \tmmathbf{e_{\theta}} \frac{1}{r} \frac{\partial}{\partial \theta} +
    \tmmathbf{e_{\varphi}} \frac{1}{R_0} \frac{\partial}{\partial \varphi}
    \right) \cdot \left( \tmmathbf{e_r} \frac{\partial \psi}{\partial r} +
    \tmmathbf{e_{\theta}} \frac{1}{r} \frac{\partial \psi}{\partial \theta} +
    \tmmathbf{e_{\varphi}} \frac{1}{R_0} \frac{\partial \psi}{\partial
    \varphi} \right) \nonumber\\
    & = & \tmcolor{red}{\tmmathbf{e_r} \frac{\partial}{\partial r} \cdot
    \left( \tmmathbf{e_r} \frac{\partial \psi}{\partial r} \right) +
    \tmmathbf{e_r} \frac{\partial}{\partial r} \cdot \left(
    \tmmathbf{e_{\theta}} \frac{1}{r} \frac{\partial \psi}{\partial \theta}
    \right) + \tmmathbf{e_r} \frac{\partial}{\partial r} \cdot \left(
    \tmmathbf{e_{\varphi}} \frac{1}{R_0} \frac{\partial \psi}{\partial
    \varphi} \right)} \nonumber\\
    & + & \tmcolor{blue}{\tmmathbf{e_{\theta}} \frac{1}{r}
    \frac{\partial}{\partial \theta} \cdot \left( \tmmathbf{e_r}
    \frac{\partial \psi}{\partial r} \right) + \tmmathbf{e_{\theta}}
    \frac{1}{r} \frac{\partial}{\partial \theta} \cdot \left(
    \tmmathbf{e_{\theta}} \frac{1}{r} \frac{\partial \psi}{\partial \theta}
    \right) + \tmmathbf{e_{\theta}} \frac{1}{r} \frac{\partial}{\partial
    \theta} \cdot \left( \tmmathbf{e_{\varphi}} \frac{1}{R_0} \frac{\partial
    \psi}{\partial \varphi} \right)} \nonumber\\
    & + & \tmmathbf{e_{\varphi}} \frac{1}{R_0} \frac{\partial}{\partial
    \varphi} \cdot \left( \tmmathbf{e_r} \frac{\partial \psi}{\partial r}
    \right) + \tmmathbf{e_{\varphi}} \frac{1}{R_0} \frac{\partial}{\partial
    \varphi} \cdot \left( \tmmathbf{e_{\theta}} \frac{1}{r} \frac{\partial
    \psi}{\partial \theta} \right) + \tmmathbf{e_{\varphi}} \frac{1}{R_0}
    \frac{\partial}{\partial \varphi} \cdot \left( \tmmathbf{e_{\varphi}}
    \frac{1}{R_0} \frac{\partial \psi}{\partial \varphi} \right) \nonumber\\
    &  &  \nonumber\\
    & = & \tmcolor{red}{(\tmmathbf{e_r \cdot} \tmmathbf{e_r})
    \frac{\partial^2 \psi}{\partial r^2} + \left( \tmmathbf{e_r \cdot}
    \frac{\partial \tmmathbf{e_r}}{\partial r} \right) \frac{\partial
    \psi}{\partial r} + (\tmmathbf{e_r \cdot} \tmmathbf{e_{\theta}})
    \frac{\partial^2 \psi}{\partial r \partial \theta} \frac{1}{r} + \left(
    e_r \cdot \frac{\partial e_{\theta}}{\partial r} \right) \frac{1}{r}
    \frac{\partial \psi}{\partial \theta}} \nonumber\\
     &+& \tmcolor{red}{(\tmmathbf{e_r \cdot}
    \tmmathbf{e_{\theta}}) \frac{\partial \psi}{\partial \theta}
    \frac{\partial}{\partial r} \left( \frac{1}{r} \right) + (\tmmathbf{e_r
    \cdot} \tmmathbf{e_{\varphi}}) \frac{1}{R_0} \frac{\partial^2
    \psi}{\partial r \partial \varphi} + \left( \tmmathbf{e_r} \cdot
    \frac{\partial \tmmathbf{e_{\varphi}}}{\partial r} \right) \frac{1}{R_0}
    \frac{\partial \psi}{\partial \varphi}} \nonumber\\
    & + & \tmcolor{blue}{(\tmmathbf{e_{\theta} \cdot} \tmmathbf{e_r})
    \frac{1}{r} \frac{\partial^2 \psi}{\partial \theta \partial r} + \left(
    \tmmathbf{e_{\theta} \cdot} \frac{\partial \tmmathbf{e_r}}{\partial
    \theta} \right) \frac{1}{r} \frac{\partial \psi}{\partial r} +
    (\tmmathbf{e_{\theta} \cdot} \tmmathbf{e_{\theta}}) \frac{\partial^2
    \psi}{\partial \theta^2} \frac{1}{r^2}} \nonumber\\
 	& + & \tmcolor{blue}{ \left( \tmmathbf{e_{\theta}}
    \cdot \frac{\partial \tmmathbf{e_{\theta}}}{\partial \theta} \right)
    \frac{1}{r^2} \frac{\partial \psi}{\partial \theta} +
    (\tmmathbf{e_{\theta} \cdot} \tmmathbf{e_{\theta}}) \frac{\partial
    \psi}{\partial \theta} \frac{1}{r} \frac{\partial}{\partial \theta} \left(
    \frac{1}{r} \right) + (\tmmathbf{e_{\theta} \cdot} \tmmathbf{e_{\varphi}})
    \frac{1}{r} \frac{1}{R_0} \frac{\partial^2 \psi}{\partial \theta \partial
    \varphi} + \left( \tmmathbf{e_{\theta}} \cdot \frac{\partial
    \tmmathbf{e_{\varphi}}}{\partial \theta} \right) \frac{1}{r} \frac{1}{R_0}
    \frac{\partial \psi}{\partial \varphi}} \nonumber\\
    & + & (\tmmathbf{e_{\varphi} \cdot} \tmmathbf{e_r}) \frac{1}{R_0}
    \frac{\partial^2 \psi}{\partial \varphi \partial r} + \left(
    \tmmathbf{e_{\varphi} \cdot} \frac{\partial \tmmathbf{e_r}}{\partial
    \varphi} \right) \frac{1}{R_0} \frac{\partial \psi}{\partial r} +
    (\tmmathbf{e_{\varphi} \cdot} \tmmathbf{e_{\theta}}) \frac{1}{R_0}
    \frac{1}{r} \frac{\partial^2 \psi}{\partial \varphi \partial \theta} +
    \left( \tmmathbf{e_{\varphi} \cdot} \frac{\partial
    \tmmathbf{e_{\theta}}}{\partial \varphi} \right) \frac{1}{R_0} \frac{1}{r}
    \frac{\partial \psi}{\partial \theta} \nonumber\\
    &+& (\tmmathbf{e_{\varphi} \cdot}
    \tmmathbf{e_{\theta}}) \frac{1}{R_0} \frac{\partial \psi}{\partial \theta}
    \frac{\partial}{\partial \varphi} \left( \frac{1}{r} \right) +
    (\tmmathbf{e_{\varphi} \cdot} \tmmathbf{e_{\varphi}}) \frac{1}{R_0^2}
    \frac{\partial^2 \psi}{\partial \varphi^2} + \left( \tmmathbf{e_{\varphi}
    \cdot} \frac{\partial \tmmathbf{e_{\varphi}}}{\partial \varphi} \right)
    \frac{1}{R_0^2} \frac{\partial \psi}{\partial \varphi} \nonumber\\
    &  &  \nonumber\\
    & = & \frac{\partial^2 \psi}{\partial r^2} + \frac{1}{r} \frac{\partial
    \psi}{\partial r} + \frac{\partial^2 \psi}{\partial \theta^2}
    \frac{1}{r^2} + \frac{\cos \theta}{R_0} \frac{\partial \psi}{\partial r} -
    \frac{\sin \theta}{R_0 r} \frac{\partial \psi}{\partial \theta} +
    \frac{1}{R_0^2} \frac{\partial^2 \psi}{\partial \varphi^2} \nonumber\\
    & = & \frac{1}{r} \frac{\partial}{\partial r} \left( r \frac{\partial
    \psi}{\partial r} \right) + \frac{1}{r^2} \frac{\partial^2 \psi}{\partial
    \theta^2} + \frac{1}{R_0^2} \frac{\partial^2 \psi}{\partial \varphi^2} +
    \frac{\cos \theta}{R_0} \frac{\partial \psi}{\partial r} - \frac{\sin
    \theta}{R_0 r} \frac{\partial \psi}{\partial \theta} 
  \end{eqnarray}
\end{enumerateroman}

\subsection{上述的2.1-2.4推导的源头}

\

上述直角坐标($x, y, z$),圆柱坐标($R, \phi, z$),球坐标($r,
\theta, \phi$),环坐标($r, \theta,
\varphi$)的推导也可以作如下的理解 。

首先,给出最一般的梯度、散度、旋度的计算公式:
\begin{eqnarray}
  \tmop{grad} \psi & = & \nabla \psi \nonumber\\
  & = & \tmmathbf{e_1} \frac{1}{h_1} \frac{\partial \psi}{\partial q_1} +
  \tmmathbf{e_2} \frac{1}{h_2} \frac{\partial \psi}{\partial q_2} +
  \tmmathbf{e_3} \frac{1}{h_3} \frac{\partial \psi}{\partial q_3} 
\end{eqnarray}
\begin{eqnarray}
  \tmop{div} \tmmathbf{f} & = & \nabla \cdot \tmmathbf{f} \nonumber\\
  & = & \frac{1}{h_1 h_2 h_3} \left[ \frac{\partial (f_1 h_2 h_3)}{\partial
  q_1} + \frac{\partial (f_2 h_1 h_3)}{\partial q_2} + \frac{\partial (f_3 h_1
  h_2)}{\partial q_3} \right] 
\end{eqnarray}
\begin{eqnarray*}
  \tmop{rot} \tmmathbf{f} & = & \nabla \times \tmmathbf{f}\\
  & = & \frac{\tmmathbf{e_1}}{h_2 h_3} \left[ \frac{\partial}{\partial q_2}
  (h_3 f_3) - \frac{\partial}{\partial q_3} (h_2 f_2) \right] +
  \frac{\tmmathbf{e_2}}{h_1 h_3} \left[ \frac{\partial}{\partial q_3} (h_1
  f_1) - \frac{\partial}{\partial q_1} (h_3 f_3) \right] +
  \frac{\tmmathbf{e_3}}{h_1 h_2} \left[ \frac{\partial}{\partial q_1} (h_2
  f_2) - \frac{\partial}{\partial q_2} (h_1 f_1) \right]
\end{eqnarray*}
\begin{eqnarray*}
  \Delta \psi & = & \nabla^2 \psi\\
  & = & \frac{1}{h_1 h_2 h_3} \left[ \frac{\partial}{\partial q_1} \left(
  \frac{h_2 h_3}{h_1} \frac{\partial \psi}{\partial q_1} \right) +
  \frac{\partial}{\partial q_2} \left( \frac{h_1 h_3}{h_2} \frac{\partial
  \psi}{\partial q_2} \right) + \frac{\partial}{\partial q_3} \left( \frac{h_1
  h_2}{h_3} \frac{\partial \psi}{\partial q_3} \right) \right]
\end{eqnarray*}


上式中的$h_1 ,h_2
,h_3$分别为相应坐标系下的拉梅因子,$q_1 ,q_2
,q_3$分别为相应坐标系下的坐标,$\tmmathbf{e_1} ,
\tmmathbf{e_2} ,
\tmmathbf{e_3}$为相应坐标系下的坐标基矢。下面说明这三个公式怎么来的,再说明相应的拉梅因子是怎么确定的。

(一)上面四个公式的由来:

曲线正交坐标系($q_1 ,q_2
,q_3$),每一点的单位正交基失($\tmmathbf{e_1} , \tmmathbf{e_2}
, \tmmathbf{e_3}$)构成右手系。

空间中任一点M在直角坐标系中是由($x, y,
z$)决定的,此时矢量r的表达式是:
\[ \tmmathbf{r =} x \tmmathbf{i} + y \tmmathbf{j} + z \tmmathbf{k} \]


但是我们也可以用另外三个数($q_1 ,q_2
,q_3$)唯一决定M点,$q_1 ,q_2
,q_3$称为曲线坐标。此时矢量$r$在曲线坐标系中的表达式为:
\[ d \tmmathbf{r} = \frac{\partial \tmmathbf{r}}{\partial q_1} \tmop{dq}_1 +
   \frac{\partial \tmmathbf{r}}{\partial q_2} \tmop{dq}_2 + \frac{\partial
   \tmmathbf{r}}{\partial q_3} \tmop{dq}_3 \]


$\frac{\partial \tmmathbf{r}}{\partial q_i}$的大小是:
\begin{eqnarray*}
  \left| \frac{\partial \tmmathbf{r}}{\partial q_1} \right| & = & \sqrt{\left(
  \frac{\partial x}{\partial q_1} \right)^2 + \left( \frac{\partial
  y}{\partial q_1} \right)^2 + \left( \frac{\partial z}{\partial q_1}
  \right)^2} = h_1\\
  \left| \frac{\partial \tmmathbf{r}}{\partial q_2} \right| & = & \sqrt{\left(
  \frac{\partial x}{\partial q_2} \right)^2 + \left( \frac{\partial
  y}{\partial q_2} \right)^2 + \left( \frac{\partial z}{\partial q_2}
  \right)^2} = h_2\\
  \left| \frac{\partial \tmmathbf{r}}{\partial q_3} \right| & = & \sqrt{\left(
  \frac{\partial x}{\partial q_3} \right)^2 + \left( \frac{\partial
  y}{\partial q_3} \right)^2 + \left( \frac{\partial z}{\partial q_3}
  \right)^2} = h_3
\end{eqnarray*}


$h_1 ,h_2
,h_3$分别为相应坐标系下的拉梅系数。考虑到$\frac{\partial
\tmmathbf{r}}{\partial q_i}$的方向后,可得:
\[ d \tmmathbf{r} = h_1 \tmop{dq}_1 \tmmathbf{e_1} + h_2 \tmop{dq}_2
   \tmmathbf{e_2} + h_3 \tmop{dq}_3 \tmmathbf{e_3} \]


这是弧元素矢量在曲线坐标系中的表达式,它们在坐标轴上的投影分别是:
\begin{eqnarray*}
  \tmop{ds}_1 & = & h_1 \tmop{dq}_1\\
  \tmop{ds}_2 & = & h_2 \tmop{dq}_2\\
  \tmop{ds}_3 & = & h_3 \tmop{dq}_3
\end{eqnarray*}


各面的侧面积元为:
\begin{eqnarray*}
  d \sigma_1 & = & h_2 h_3 \tmop{dq}_2 \tmop{dq}_3\\
  d \sigma_2 & = & h_1 h_3 \tmop{dq}_1 \tmop{dq}_3\\
  d \sigma_3 & = & h_1 h_2 \tmop{dq}_1 \tmop{dq}_2
\end{eqnarray*}


体积元为:
\[ \tmop{dv} = h_1 h_2 h_3 \tmop{dq}_1 \tmop{dq}_2 \tmop{dq}_3 \]


下面分别求梯度、散度、旋度
\begin{itemize}

  \item
  梯度:由梯度的定义,梯度等于各个方向的方向导数乘以该方向的单位矢量。
  \begin{equation}
    \nabla \psi = \tmop{grad} \psi = \frac{1}{h_1} \frac{\partial
    \psi}{\partial q_1} \tmmathbf{e_1} + \frac{1}{h_2} \frac{\partial
    \psi}{\partial q_2} \tmmathbf{e_2} + \frac{1}{h_3} \frac{\partial
    \psi}{\partial q_3} \tmmathbf{e_3}
  \end{equation}
  \item 散度:先证明散度的定义:
  
  在直角坐标系下的高斯公式为:
% \[ \bigiiint_v (\nabla \cdot \tmmathbf{f}) \tmop{dV} = \bigoiint_s     \tmmathbf{f} \cdot d \tmmathbf{S} \]

\[\iint_{D}f(x,y)dxdy 
\
 \iiint_{\Omega}f(x,y,z)dxdydz  \]

\begin{equation}
\oiint \limits_{S} \mathbf{A} \cdot \mathrm{d}\mathbf{S} = 
\iiint \limits_{V} \nabla\cdot \mathbf{A} \mathrm{d}V
\end{equation}


\begin{equation}
\iint_{D} 2xy\mathrm{d}x\mathrm{d}y
\end{equation}


  换成曲线坐标系下可得:
  

\end{itemize}


(二)拉梅系数的求解:有两种,第一种是由微分几何中的度规来求,第二中是利用弧长等价来求,实际上,第二种属于第一种。只是现在我不知道什么是度规,所以先用第二种来求。
\begin{enumeratealphacap}
  \item 柱坐标系($\rho, \theta, z$)下的拉梅系数:
  
  柱坐标系($\rho, \theta, z$)与直角坐标系($x, y,
  z$)的变换为:
  \begin{eqnarray*}
    x & = & \rho \cos \theta\\
    y & = & \rho \sin \theta\\
    z & = & z
  \end{eqnarray*}
  直角坐标的全微分为:
  \begin{eqnarray*}
    \tmop{dx} & = & \cos \theta d \rho - \tmop{psin} \theta d \theta\\
    \tmop{dy} & = & \sin \theta d \rho + \rho \cos \theta d \theta\\
    \tmop{dz} & = & \tmop{dz}
  \end{eqnarray*}
  微元弧长:
  \begin{eqnarray*}
    (\tmop{ds})^2 & = & (\tmop{dx})^2 + (\tmop{dy})^2 + (\tmop{dz})^2\\
    & = & (\cos \theta d \rho - \tmop{psin} \theta d \theta)^2 + (\sin \theta
    d \rho + \rho \cos \theta d \theta)^2 + (\tmop{dz})^2\\
    & = & (d \rho)^2 + (\rho d \theta)^2 + (\tmop{dz})^2
  \end{eqnarray*}
  所以:
  \begin{eqnarray*}
    h_1 & = & 1\\
    h_2 & = & \rho\\
    h_3 & = & 1
  \end{eqnarray*}
  \item 求坐标系($r, \theta, \phi$)下的拉梅系数
  
  柱坐标系($r, \theta, \phi$)与直角坐标系($x, y,
  z$)的变换为:
  \begin{eqnarray*}
    x & = & \tmop{rsin} \theta \cos \phi\\
    y & = & \tmop{rsin} \theta \sin \phi\\
    z & = & \tmop{rcos} \theta
  \end{eqnarray*}
  直角坐标的全微分为:
  \begin{eqnarray*}
    \tmop{dx} & = & \sin \theta \cos \phi \tmop{dr} + \tmop{rcos} \theta \cos
    \phi d \theta - \tmop{rsin} \theta \sin \phi d \phi\\
    \tmop{dy} & = & \sin \theta \sin \phi \tmop{dr} + \tmop{rcos} \theta \sin
    \phi d \theta + \tmop{rsin} \theta \cos \phi d \phi\\
    \tmop{dz} & = & \cos \theta \tmop{dr} - \tmop{rsin} \theta d \theta
  \end{eqnarray*}
  微元弧长:
  \begin{eqnarray*}
    (\tmop{ds})^2 & = & (\tmop{dx})^2 + (\tmop{dy})^2 + (\tmop{dz})^2\\
    & = & (\sin \theta \cos \phi \tmop{dr} + \tmop{rcos} \theta \cos \phi d
    \theta - \tmop{rsin} \theta \sin \phi d \phi)^2 + (\sin \theta \sin \phi
    \tmop{dr} + \tmop{rcos} \theta \sin \phi d \theta + \tmop{rsin} \theta
    \cos \phi d \phi)^2 + (\cos \theta \tmop{dr} - \tmop{rsin} \theta d
    \theta)^2\\
    & = & (\tmop{dr})^2 + (\tmop{rd} \theta)^2 + (\tmop{rsin} \theta d
    \phi)^2
  \end{eqnarray*}
  以:
  \begin{eqnarray*}
    h_1 & = & 1\\
    h_2 & = & r\\
    h_3 & = & \tmop{rsin} \theta
  \end{eqnarray*}
\end{enumeratealphacap}

\section{张量及其分析}


\begin{enumeratenumeric}
  \item 矢量与张量点乘,结果为矢量
  \begin{eqnarray*}
    \left( \tmmathbf{f \cdot \overset{\longleftrightarrow \text{}}{T}}
    \right)_j & = & f_i T_{\tmop{ij}}\\
    \left( \tmmathbf{\overset{\longleftrightarrow \text{}}{T} \cdot f}
    \right)_j & = & T_{\tmop{ji}} f_i
  \end{eqnarray*}
  矢量与张量点乘不具有互换性,即一般
  \[ \tmmathbf{f \cdot \overset{\longleftrightarrow \text{}}{T}} \neq
     \tmmathbf{\overset{\longleftrightarrow \text{}}{T} \cdot f} \]
  但是当$\overset{\longleftrightarrow
  \text{}}{\tmmathbf{T}}$为对称张量时,$\tmmathbf{f \cdot
  \overset{\longleftrightarrow \text{}}{T}} =
  \tmmathbf{\overset{\longleftrightarrow \text{}}{T} \cdot
  f}$成立,特别地,任何矢量与单位张量$\overset{\longleftrightarrow}{\tmmathbf{I}}$之间的点乘等于矢量本身:$\tmmathbf{f
  \cdot \overset{\longleftrightarrow \text{}}{I}} =
  \tmmathbf{\overset{\longleftrightarrow \text{}}{I} \cdot
  f}$=\tmtextbf{$\tmmathbf{f}$}。作为矢量与张量点乘的特例,矢量与并矢点乘结果也为矢量,即:
  \[ \tmmathbf{f \cdot (\tmop{gh}) = (f \cdot g) h} \]
  \[ \tmmathbf{(\tmop{gh}) \cdot f} = \tmmathbf{g (h \cdot f)} \]
  注意,矢量\tmtextbf{$\tmmathbf{f}$}总是与并矢中与之相邻的矢量点乘,交换参与点乘的矢量和并矢的顺序,一般将导致不同的结果。一般情况下并矢的两个矢量交换位置后的结果不同:
  \[ \tmmathbf{\tmop{gh} \neq \tmop{hg}} \]
  \item 张量(或并矢)之间的点乘,结果为张量,即:
  \begin{eqnarray*}
    \left( \overset{\longleftrightarrow}{\tmmathbf{S}} \tmmathbf{\cdot}
    \overset{\longleftrightarrow}{\tmmathbf{T}} \right)_{\tmop{ij}} & = &
    S_{\tmop{ik}} T_{\tmop{kj}}\\
    \left( \overset{\longleftrightarrow}{\tmmathbf{T}} \tmmathbf{\cdot}
    \overset{\longleftrightarrow}{\tmmathbf{S}} \right)_{\tmop{ij}} & = &
    T_{\tmop{ik}} S_{\tmop{kj}}\\
    (\tmmathbf{\tmop{fg}}) \tmmathbf{\cdot (\tmop{pq})} & = & (\tmmathbf{g
    \cdot p}) \tmmathbf{\tmop{fq}}\\
    (\tmmathbf{\tmop{pq}}) \tmmathbf{\cdot} (\tmmathbf{\tmop{fg}}) & = &
    (\tmmathbf{q \cdot f}) \tmmathbf{\tmop{pg}}
  \end{eqnarray*}
  规则如下:前导张量的第二指标(列标)与后随张量的第一指标(行标)为求和指标,前导并矢的第二个矢量和后随并矢的第一矢量之间实施点乘,张量点乘不具有交换性除非其中一个为单位张量:
  \begin{eqnarray*}
    \overset{\longleftrightarrow}{\tmmathbf{S}} \tmmathbf{\cdot}
    \overset{\longleftrightarrow}{\tmmathbf{T}} & \neq &
    \overset{\longleftrightarrow}{\tmmathbf{T}} \tmmathbf{\cdot}
    \overset{\longleftrightarrow}{\tmmathbf{S}}\\
    \overset{\longleftrightarrow}{\tmmathbf{S}} \tmmathbf{\cdot}
    \overset{\longleftrightarrow}{\tmmathbf{I}} & = &
    \overset{\longleftrightarrow}{\tmmathbf{I}} \tmmathbf{\cdot}
    \overset{\longleftrightarrow}{\tmmathbf{S}} =
    \overset{\longleftrightarrow}{\tmmathbf{S}}
  \end{eqnarray*}
  \item 张量与张量双点乘,结果为标量,即:
  \begin{eqnarray*}
    \overset{\longleftrightarrow}{\tmmathbf{S}} :
    \overset{\longleftrightarrow}{\tmmathbf{T}} & = &
    \overset{\longleftrightarrow}{\tmmathbf{T}} :
    \overset{\longleftrightarrow}{\tmmathbf{S}} = S_{\tmop{ij}} T_{\tmop{ji}}
  \end{eqnarray*}
  双点乘具有交换性。任何张量与单位张量$\overset{\longleftrightarrow}{\tmmathbf{I}}$之间的双点乘等于张量的迹(张量对角元素之和):
  \[ \overset{\longleftrightarrow}{\tmmathbf{T}} \tmmathbf{:}
     \overset{\longleftrightarrow}{\tmmathbf{I}} =
     \overset{\longleftrightarrow}{\tmmathbf{I}} \tmmathbf{:}
     \overset{\longleftrightarrow}{\tmmathbf{T}} = \tmop{Tr} \left(
     \overset{\longleftrightarrow}{\tmmathbf{T}} \right) \]
  两单位张量的双点乘等于3,两并矢的双点乘为:
  \[ \begin{array}{lll}
       (\tmmathbf{\tmop{fg}}) \tmmathbf{:} (\tmmathbf{\tmop{pq}}) & = &
       (\tmmathbf{g \cdot p}) (\tmmathbf{f \cdot q})
     \end{array} \]
  \item 矢量与张量的叉乘和张量与矢量的叉乘,结果为张量
  \begin{eqnarray*}
    \overset{\longleftrightarrow}{\tmmathbf{S}} & = & \tmmathbf{f \times}
    \overset{\longleftrightarrow}{\tmmathbf{T}}\\
    S_{\tmop{ij}} & = & \varepsilon_{\tmop{imn}} f_m T_{\tmop{nj}}\\
    \overset{\longleftrightarrow}{\tmmathbf{R}} & = &
    \overset{\longleftrightarrow}{\tmmathbf{T}} \times \tmmathbf{f}\\
    R_{\tmop{ij}} & = & \varepsilon_{\tmop{jmn}} T_{\tmop{im}} f_n
  \end{eqnarray*}
  上述运算不具有交换性,即一般$\overset{\longleftrightarrow}{\tmmathbf{R}}
  \neq
  \overset{\longleftrightarrow}{\tmmathbf{S}}$,除非$\overset{\longleftrightarrow}{\tmmathbf{T}}$为单位张量,此时有:
  \begin{eqnarray*}
    \tmmathbf{f \times} \overset{\longleftrightarrow}{\tmmathbf{I}} & = &
    \overset{\longleftrightarrow}{\tmmathbf{I}} \times \tmmathbf{f}\\
    & = & \left(\begin{array}{ccc}
      0 & - f_3 & f_2\\
      f_3 & 0 & - f_1\\
      - f_2 & f_1 & 0
    \end{array}\right)
  \end{eqnarray*}
  张量为并矢的特殊情况,掌握就近叉乘的法则:
  \begin{eqnarray*}
    \tmmathbf{f \times (\tmop{gh})} & = & (\tmmathbf{f \times g})
    \tmmathbf{h}\\
    (\tmmathbf{\tmop{gh}}) \times \tmmathbf{f} & = & \tmmathbf{g (h \times f)}
  \end{eqnarray*}
  结果为并矢。
  
  \item
  算符$\nabla$作用在张量或并矢上,只要注意$\nabla$算符的矢量特性和微分运算特性,可以得出:
  \begin{eqnarray*}
    \nabla \tmmathbf{\cdot} (\tmmathbf{\tmop{fg}}) & = & (\nabla
    \tmmathbf{\cdot f} \tmmathbf{}) \tmmathbf{g} + (\tmmathbf{f} \cdot \nabla)
    \tmmathbf{g}\\
    \nabla \tmmathbf{\cdot} \overset{\longleftrightarrow}{\mathfrak{I}} & = &
    \frac{\partial}{\partial x} \left( \tmmathbf{e_x} \tmmathbf{\cdot}
    \overset{\longleftrightarrow}{\mathfrak{I}} \right) +
    \frac{\partial}{\partial y} \left( \tmmathbf{e_y} \tmmathbf{\cdot}
    \overset{\longleftrightarrow}{\mathfrak{I}} \right) +
    \frac{\partial}{\partial z} \left( \tmmathbf{e_z} \tmmathbf{\cdot}
    \overset{\longleftrightarrow}{\mathfrak{I}} \right)
  \end{eqnarray*}
  上述两式的结果都是矢量。
\end{enumeratenumeric}

\section{高斯公式和斯托克斯}

\

1.高斯公式:

\begin{equation}
	\oiint_{s} \tmmathbf{A \cdot} d \tmmathbf{S} = \oiint_{v} \nabla
	\tmmathbf{\cdot} \tmmathbf{A} \tmop{dV} 
\end{equation} 


式中,$v$为封闭曲面$s$所维的体积。

\

2.斯托克斯公式
\begin{equation}
\oint_{L} \tmmathbf{A \cdot} d \tmmathbf{l} = \iint_s (\nabla \times
\tmmathbf{A}) \tmmathbf{\cdot} d \tmmathbf{S} 
\end{equation} 

式中,$S$为以$L$闭曲线为边界的曲面,且L的正向与$S$曲面的法线方向(即$d
\tmmathbf{S}$的方向)遵循右手定则。

\


\end{document}
